\documentclass[a4paper, 12pt]{extarticle}
\usepackage[utf8]{inputenc}
\usepackage[T2A]{fontenc}
\usepackage[english,russian]{babel}
\usepackage{amsmath}

\usepackage{geometry}
\geometry{left=3cm}
\geometry{right=1.5cm}
\geometry{top=2cm}
\geometry{bottom=2cm}

\geometry{marginparwidth=2.5cm}
\newcommand{\fnt}[1]{\fontsize{#1}{\baselineskip}\selectfont}
\begin{document}
\thispagestyle{empty}
\begin{center}
\textbf{Министерство науки и высшего образования Российской Федерации\\Федеральное государственное автономное образовательное\\учреждение высшего образования «Новосибирский национальный\\исследовательский государственный университет»\\ Физический факультет}
\end{center}

{\setlength{\parindent}{0cm}
Кафедра ФИЗИКО-ТЕХНИЧЕСКОЙ ИНФОРМАТИКИ\\
Направление подготовки 03.04.02 ФИЗИКА
}
\\~
\begin{center}
    \textbf{Аннотация\\
    к выпускной квалификационной работе}\\
    Андреева Андрея Андреевича\\
    <<Блок упаковки данных для системы FEX модуля LASP жидкоаргонового калориметра детектора ATLAS>>
\end{center}
\par
Эксперимен ATLAS – один из четырёх основных на Большом адронном коллайдере (БАК). Эксперимент проводится на одноимённом детекторе, предназначенном для исследования протон-протонных столкновений и столкновений тяжелых ионов. С помощью детектора ATLAS регистрируются данные, которые используются для дальнейшего изучения свойств бозона Хиггса, поиска суперсимметричных частиц и широкого набора других задач. Также оним из важнейших направлений исследований являются попытки обнаружить физические явления, лежащие за границами Стандартной Модели.\par
Одной из основных подсистем детектора ATLAS является система жидкоаргоновых калориметров. В рамках второй фазы её модернизации ведётся проектирование совершенно новой системы считывающей электроники, которая будет установлена в период третьего длительного отключения БАК (2024 -- 2026 гг.), что позволит существенно повысить мгновенную светимость до $7,5*10^{34}\text{см}^{-2}\text{с}^{-1}$. Важным компонентом новой считывающей электроники является модуль сигнального процессора LASP (Liquid Argon Signal Processor), с помощью которого реализуется первичная цифровая обработка оцифрованных данных. LASP проектируется на базе микросхем программируемой логики (ПЛИС).\par
В ходе данной работы велась разработка блока упаковки данных для системы FEX (Feature EXtractor) сигнального процессора LASP, в состав которого входят модули конфигурируемой перестановки remap и упаковщик триггерных данных packer. В результате для модуля remap были реализованы и встроены в общий проект сигнального процессора синтезируемые блоки цифровой логики, подготовлено симуляционно окружение для моделирования его поведения и проведена компиляция под целевую аппаратную платформу. Кроме того, создан прототип программного обеспечения для автоматической генерации конфигураций перестановок для восстановления порядка данных в соответствии с геометрией детектора. Также разработан набор вариантов протокола упаковки данных в кадры для системы fFEX (forward Feature EXtractor) и синтезируемый прототип модуля упаковки триггерных данных packer для этой системы. Реализация блоков цифровой аппаратуры велась с помощью языка описания логических схем VHDL.
\\~\\~\\
{\setlength{\parindent}{0cm}
$\underline{\hspace{4cm}}\text{/}\underline{\hspace{4cm}}$\\~\\
<<$\underline{\hspace{1cm}}$>>$\underline{\hspace{4cm}}$ 2022 г.\\
}
\end{document}

