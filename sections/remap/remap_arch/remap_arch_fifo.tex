Второй вариант архитектуры компонента Remap, содержащий память FIFO, представлен на рисунке \ref{fig:remap_fifo}.\par
\begin{figure}[ht]
    \centering
    \includegraphics[width=\linewidth]{remap_fifo.png}
    \caption{Схема архитектуры модуля Remap, основанной на FIFO ПЕРЕВЕСТИ}
    \label{fig:remap_fifo}
\end{figure}\par
В рамках данного подхода в качестве буфера для мультиплексированных данных является память FIFO (First In First Out). Такая структура состоит из двухпортовой памяти, двух счётчиков адреса и двух автоматов для чтения и записи данных и является одним из ключевых элементов цифровой схемотехники. Одно из самых распространённых применений такой памяти, помимо буферизации информации -- это реализация перехода данных между тактовыми доменами. Поскольку такая память используется невероятно часто в проектировании логических схем, то существует множество готовых вариантов их реализации, в том числе и от разработчиков самих микросхем ПЛИС и соответствующего программного обеспечения для автоматического проектирования, в том числе и от Intel. В случае использования такого готового блока FIFO не требуется ручное написание временных ограничений, что избавляет от потенциальных проблем на этапе синтеза цифровой схемы всего проекта.\par
Однако, одна из основных особенностей FIFO -- это сохранение порядка записываемых данных, что не позволяет реализовать последний этап работы модуля Remap. Для решения данной задачи используется подход, при котором данные с мультиплексора поступают не напрямую на вход FIFO, а записываются в один большой регистр, достаточного размера для одновременного хранения всех мультиплексированных данных в рамках текущего столкновения пучков. Для наиболее оптимального использования логических ресурсов этот регистр является сдвиговым, то есть каждый такт новое значение поступает в начало, после чего оно смещается дальше. Только после полного заполнения этого регистра актуальными величинами, данные одним большим словом записывается в FIFO. Считывающая логика, после обнаружения данных на выходе FIFO, имеет доступ сразу ко всем значениям и может извлекать их последовательно в необходимом порядке.\par
В целях минимизации латентности необходимо, чтобы поступающие в FIFO данные сразу же были доступны для чтения, то есть требуется не допускать его заполнения. Поскольку запись и извлечение идёт с одной и той же скоростью, важно важно сделать так, чтобы считывающая система начала работу как минимум не позднее записывающей. Это достигается правильным управлением сигналами сброса: после старта сигнального процессора LASP сначала должен сняться сброс, синхронный с тактовым доменом $f_{core}$, а уже затем $f_{feb}$.\par
