Модуль Remap является частью жидкоаргонового сигнального процессора LASP (рис. \ref{fig:remap_lasp}) и в первую очередь предназначен для организации упорядочивания входных данных в соответствии с геометрией детектора.\par
\begin{figure}[ht]
    \centering
    \includegraphics[width=0.8\linewidth]{remap_lasp.png}
    \caption{Схема расположения модуля Remap в общей структуре сигнального процессора LASP}
    \label{fig:remap_lasp}
\end{figure}\par
Remap компонент должен реализовывать приём поступающей информации и формировать из неё набор выходных каналов данных, в каждом из которых обязаны передаваться значения АЦП, выбранные из входных каналов в соответствии с установленной конфигурацией, в определённом порядке, также согласно конфигурации.\par

\subsection{Архитектура модуля}
Основополагающим подходом в проектировании модуля конфигурируемой перестановки Remap является создание таких элементарных устройств, которые способны принимать на вход весь требуемый объём данных и формировать из него лишь один выходной канал. Это обеспечивает высокую гибкость в масштабировании, поскольку в таком случае реализация необходимого количества выходных каналов достигается простой репликацией подобных структур, как это показано на рисунке \ref{fig:remap_replication}.\par

\begin{figure}[ht]
    \centering
    \includegraphics[width=\linewidth]{remap_replication.png}
    \caption{Схема формирования необходимого количества выходных каналов ПЕРЕРИСОВАТЬ}
    \label{fig:remap_replication}
\end{figure}\par

Стоит отметить, поскольку входной поток данных разбивается на 4 независимые части, обрабатывающиеся отдельно, каждый модуль Remap должен принимать лишь 22 канала и формировать 16 или 11 выходов, в зависимости от варианта сигнального процессора LASP.\par
В ходе разработки было спроектировано и реализовано два различных варианта архитектуры базового элемента модуля Remap: с модулем синхронизации тактовых доменов, а также архитектура, основанная на FIFO памяти.\par

\subsubsection{Архитектура с модулем синхронизации тактовых доменов}
Первый вариант архитектуры компонента Remap, содержащий специальный модуль синхронизации тактовых доменов, представлен на рисунке \ref{fig:remap_cds}.\par
\begin{figure}[ht]
    \centering
    \includegraphics[width=\linewidth]{remap_cds.png}
    \caption{Схема архитектуры модуля Remap с модулем синхронизации тактовых доменов ПЕРЕВЕСТИ}
    \label{fig:remap_cds}
\end{figure}\par
Основной особенностью этой архитектуры является то, что в качестве буфера для мультиплексированных данных используется блок двухпортовой RAM памяти. Эта память разбита на несколько страниц, каждая из которых имеет размер, достаточный для хранения захваченной информации, относящейся к одному столкновению пучков. Чтение данных из страницы начинается лишь только после её полного заполнения записывающей стороной. Для синхронизации процессов считывания и записи предусмотрен следующий механизм: по завершению заполнения страницы памяти записывающая логика генерирует импульс шириной в один такт и отправляет его на вход специального модуля. Внутри этого модуля расположены два счётчика, работающие на тактовых частотах $f_{feb}$ и $f_{core}$, которые ведут счёт в диапазоне количества временных ячеек для каждого BCID. В рабочем режиме первый настроен так, чтобы обнуляться одновременно с поступлением синхросигнала от записывающей логики, а второй с задержкой около такта $f_{feb}$ после первого. При завершении цикла работы второго счётчика формируется выходной сигнал синхронизации, который поступает к считывающей логике и означает, что очередная страница в двухпортовой памяти заполнена полностью и можно безопасно извлекать из неё данные. Если вдруг синхронизация собьётся и синхросигнал от системы записи придёт не вовремя, то модуль это обнаружит и перейдёт в режим восстановления синхронизации. Часть данных после сбоя синхронизации будет утеряно, но через некоторое время система автоматически восстановится и продолжит работать исправно.\par
Описанная архитектура была реализована на языке описания цифровой логики VHDL и отлажена. По результатам тестирования в симуляторе она подтвердила свою работоспособность. Однако такой подход имеет ряд недостатков, главным из которых является необходимость передавать целый набор сигналов(такие как номер текущей страницы, индекс столкновения пучка, а также ряд вспомогательных сигналов внутри модуля синхронизации тактовых доменов) между тактовыми доменами $f_{feb}$ и $f_{core}$ вручную, используя схемы на двух регистрах. Для корректной организации таких переходов требуется тонкая ручная настройка временных ограничений, реализуемая путём составления специальных указаний синтезатору физической схемы, входящему в состав программного комплекса Intel Quartus Prime. Это значительно усложняет весь проект и делает его гораздо менее гибким. После возникновения проблем с разводимостью логики проекта LATOME, который является основой задетекторной электроники эксперимента ATLAS, разработанной в рамках предшествующей фазы обновления детектора, командой разработчиков сигнального процессора LASP было принято решение максимально избегать подобные способы перехода между тактовыми доменами. Кроме того, данный вариант является довольно путанным и сложным для понимания в деталях. Учитывая все эти недостатки, было решено разработать альтернативную архитектуру модуля Remap.\par


\subsubsection{Архитектура, основанная на FIFO}
Второй вариант архитектуры компонента Remap, содержащий память FIFO, представлен на рисунке \ref{fig:remap_fifo}.\par
\begin{figure}[ht]
    \centering
    \includegraphics[width=\linewidth]{remap_fifo.png}
    \caption{Схема архитектуры модуля Remap, основанной на FIFO ПЕРЕВЕСТИ}
    \label{fig:remap_fifo}
\end{figure}\par



\subsection{Конфигурирование через интерфейс медленного контроля}
Конфигурирование модуля Remap осуществляется через интерфейс медленного контроля. Как упоминалось ранее, он функционирует поверх протокола Avalon Memory Mapped, который предназначен для работы с адресуемой памятью. Такой подход очень удобен, поскольку в этом случае можно выделить каждому модулю свой участок адресов, по которым можно будет располагать необходимые значения. Разные адреса можно настроить по способу доступа к ним, таким образом можно завести некоторые показатели системы, которые можно будет только считывать, или же добавить параметры с опцией модификации. Отдельная важная особенность работы через память -- возможность функционирования в разных тактовых доменах, для этого достаточно использовать модули двухпортовой памяти. Это позволяет использовать достаточно низкую тактовую частоту для интерфейса конфигурации, чтобы он не оказывал существенного влияния на разводимость остальной логики. Причём эта частота может быть единой для конфигурирования всех компонентов, вне зависимости от их внутренних тактовых сигналов, что значительно упрощает работу медленного контроля.\par
Модуль перестановки Remap имеет две конфигурируемые стадии: какие значения извлекать из общего потока данных с помощью мультиплексора и в каком порядке их выдавать в выходной канал. Поскольку эти стадии работают в разных тактовых доменах, то необходимо размещать параметры для них в разных блоках памяти, чтобы можно было корректно переводить значения в целевые тактовые частоты. Начальный адрес конфигурации мультиплексора устанавливается глобальной константой REMAP\_BADDR(Remap Base Address) с уровня всего проекта сигнального процессора LASP, а конфигурация порядка выходных данных имеет некоторое смещение относительно него. На рисунке \ref{fig:remap_sctrl_mapping} изображена схема отображения конфигураций на адресное пространство.\par
\begin{figure}[ht]
    \centering
    \includegraphics[width=\linewidth]{remap_sctrl_mapping.png}
    \caption{Схема маппинга памяти модуля перестановки Remap для записи конфигурации}
    \label{fig:remap_sctrl_mapping}
\end{figure}\par
Для конфирурирования входного мультиплексора необходимо для каждой временной ячейки установить номер канала, с которого необходимо захватить данные. На каждый Remap поступает по 22 канала, то есть требуется 5 бит на значение. Для любого столкновения пучков выделяется по 8 временных интервалов, следовательно суммарно должно быть не менее 40 бит данных для конфигурирования одного выходного канала Remap. Шина данных интерфейса Avalon Memory Mapped имеет ширину 32 бита, поэтому для удобства формирования и чтения конфигурационных данных используется 2 слова AVMM, что составляет 64 бита. В случае варианта быстрой опции сигнального процессора LASP требуется два входных мультиплексора, соответственно размер конфигурации удваивается и равняется 128 бит.\par
Конфигурирование финальной перестановки осуществляется путём последовательного указания индекса необходимого значения. В зависимости от медленной или быстрой опции отобранных величин может быть 8 или 16 соответственно. Для более удобной работы под каждое такое значение выделяется по 4 бита. Далее, в зависимости от варианта сигнального процессора LASP требуется от 8 до 12 временных ячеек для каждого BCID, следовательно суммарно необходимо иметь от 32 до 48 бит. Аналогично конфигурации мультиплексора, в целях повышения удобства размер конфигурации округляется по ширине шины интерфейса AVMM и составляет 64 бита независимо от опции сигнального процессора LASP.\par


\subsection{Реализация}
В ходе реализации синтезируемых компонентов модуля Remap активно использовалось тестирование с помощью симуляции. Оно осуществлялось с помощью специализированного программного обеспечения Mentor QuestaSim, предназначенное для моделирования и отладки микросхем ПЛИС. Симуляционное окружение разработано, как и синтезируемые модули, на языке VHDL и обеспечивает поступление данных на входной интерфейс тестируемого модуля. Так, на рисунке \ref{fig:sim_input} приведён фрагмент симуляции, на котором показан пример данных внутри внутри  входного интерфейса. Можно увидеть, что как и в реальной системе, в каждый модуль Remap поступает 22 канала со значениями АЦП, причем для каждого BCID передаётся по 8 величин в канале. Все сигналы входного сигнала синхронны с тактовой частотой $f_{feb}$.\par
\begin{figure}[ht]
    \centering
    \includegraphics[width=\linewidth]{sim_input.png}
    \caption{Фрагмент поступающих в модуль Remap входных данных в симуляции ВЗЯТЬ МАСШТАБ ПОКРУПНЕЙ}
    \label{fig:sim_input}
\end{figure}\par
В рассматриваемом примере модуль предназначен для работы в варианте сигнального процессора LASP с установленной медленной опцией. В качестве конфигурации производится установка параметров для первых двух выходных каналов Remap компонента. На рисунке \ref{fig:sim_sctrl} отображено, как это осуществляется через интерфейс медленного контроля. На волновой диаграмме отчетливо видно, как значения поступают в установленном формате по протоколу AVMM, после чего лишние биты отсекаются, а сами конфигурационные данных переходят в соответствующие им тактовые домены. В соответствии с настройкой, первый выходной канал должен выдавать данные из первых восьми входных каналов в обратной последовательности, а второй по четыре значения из каналов с номерами 20 и 21 в чередующейся последовательности.\par
\begin{figure}[ht]
    \centering
    \includegraphics[width=\linewidth]{sim_sctrl.png}
    \caption{Пример записи конфигурации модуля Remap в симуляции}
    \label{fig:sim_sctrl}
\end{figure}\par
На рисунке \ref{fig:sim_output} изображен выходной интерфейс модуля конфигурируемой перестановки. Поскольку система предназначена для работы в медленной опции сигнального процессора LASP, выходной интерфейс состоит из 16 каналов, в котором данные передаются синхронно частоте $f_{core}$, равной 320 МГц. На нём можно отследить корректность работы компонента, работающего в соответствии с вышеописанными настройками. \par
\begin{figure}[ht]
    \centering
    \includegraphics[width=\linewidth]{sim_output.png}
    \caption{Фрагмент выходящих из модуля Remap данных в симуляции ВЗЯТЬ МАСШТАБ ПОКРУПНЕЙ}
    \label{fig:sim_output}
\end{figure}\par

