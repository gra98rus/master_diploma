После отладки функциональности в симуляторе был произведён синтез модуля \texttt{remap} под платформу КАКАЯ в обоих вариантах конфигурации сигнального процессора LASP. В таблице \ref{tab:remap_util} представлены значения используемых логических ресурсов для синтеза модуля, рассчитанного под обработку 22 входных каналов данных. Количество выходных каналов, в свою очередь, зависит от опции LASP.\par
\begin{table}[ht]
    \caption{Используемые модулем remap логические ресурсы}
    \begin{tabular}{|p{0.3\textwidth}|p{0.3\textwidth}|p{0.3\textwidth}|}
        \hline
        Тип & Медленная опция & Быстрая опция \\
        \hline
        ALM & 3077 & 2249 \\
        \hline
        ALUT & 1728 & 1686 \\
        \hline
        Регистры & 7075 & 4530 \\
        \hline
        block mem bits & 7168 & 2600 \\
        \hline
        Блоки памяти M20K & 76 & 37 \\
        \hline
    \end{tabular}
    \label{tab:remap_util}
\end{table}
По таблице отчетливо видно, что модуль использует меньше логических ресурсов ПЛИС при работе в быстрой опции, поскольку в таком варианте необходимо генерировать меньшее количество выходных каналов, но на более высокой частоте. С точки зрения временных ограничений, обе версии модуля \texttt{remap} синтезируются без проблем и имеют запас по установке сигнала в 350 и 260 пикосекунд соответственно. Задержка при передаче данных составляет порядка 30 наносекунд, что удовлетворяет требованиям, предъявляемых модулю.\par

