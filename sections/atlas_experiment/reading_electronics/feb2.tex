Платы FEB2 принимают сигналы от калориметрических ячеек и выполняют их аналоговую обработку, включая усиление, формирование и разделение на две перекрывающиеся шкалы линейного усиления. Обе шкалы усиления оцифровываются при помощи аналого-цифрового преобразователя (АЦП), после чего цифровые сигналы сериализуются и отправляются через оптический канал связи. Для этого используется несколько специализированных интегральных микросхем, а также системы управления и синхронизации. Оцифровка данных производится на частоте 40 МГц, равной частоте столкновения частиц. Каждая плата FEB2 способна обрабатывать 128 калориметрических каналов, а для считывания всей системы жидкоаргоновых калориметров требуется 1524 таких устройства.\par
Аналоговая обработка данных выполняется в 2 этапа. На первом этапе выполняется усиление сигналов калориметра, которые имеют динамический диапазон до 16 бит, с помощью специального предусилителя. Второй каскад -- формирователь, который преследует две цели. Во-первых, он необходим для преобразования выходного сигнала схемы предварительного усилителя в дифференциальный выходной сигнал с несколькими коэффициентами усиления, а во-вторых, для получения по крайней мере одного этапа формирования в соответствии с требованиями к обработке сигнала. При необходимости могут быть добавлены несколько эквивалентных этапов формирования с минимальными затратами энергии. Как предусилитель, так и формирователь реализуются в одной специализированной интегральной микросхеме LAPAS (Liquid Argon Preamplifier And Shaper \parencite{lapas}), способной обрабатывать 4 либо 8 калориметрических сигналов.\par
В дополнение к усилению и формированию сигнала необходимы периферийные схемы, такие как генератор тестовых импульсов, схема смещения, датчик температуры, а также регистры конфигурации всего модуля.\par
Далее аналоговый сигнал от каждой калориметрической ячейки оцифровывается с частотой 40 МГц, синхронно с частотой соударения пучков в Большом Адронном коллайдере. Для охвата 16-битного динамического диапазона сигнал оцифровывается с двумя шкалами усиления с помощью 14-битных АЦП. Затем каждый выходной сигнал АЦП форматируется в 16-битное слово и сериализуется со скоростью передачи данных 640 Мбит/с. Каждое такое слово помимо 14 бит данных АЦП содержит бит чётности для обеспечения проверки ошибок. Учитывая, что каждая плата FEB2 обрабатывает 128 калориметрических каналов, результирующая скорость передачи данных составляет 163,84 Гбит/с (256 потоков по 640 Мбит/с каждый). Для передачи оцифрованных данных используются специально разработанные радиационно-стойкие трансивер и лазер lpGBT (low power GigaBit Tranceiver \parencite{lpgbt}). \par
Для реализации корректной синхронизации данных калориметра в модуле FEB2 предусмотрена генерация идентификатора соударения пучков (BCID -- Bunch Crossing Identifier). Данный идентификатор представлен в виде 12-битного счётчика, который инкриминируется с частотой возникновения событий в коллайдере и сбрасывается после каждого завершения цикла столкновений пучков частиц на орбите. Значение BCID, как и данные АЦП, сериализуются и передаются в систему задетекторной электроники через оптический канал.\par
Кроме основного тракта данных в модуле FEB2 присутствует подсистема, которая обеспечивает формирование входных данных для платы LTDB (LAr Trigger Digitizer Board \parencite{ltdb}). Данная плата обрабатывает аналоговые суммы сигналов для максимально быстрого принятия решения триггерной системы, но с более грубой детализацией, чем обеспечивается основным считыванием. Модуль FEB2 имеет набор сумматоров, которые формируют требуемые аналоговые сигналы сумм по соседним ячейкам калориметра.\par
