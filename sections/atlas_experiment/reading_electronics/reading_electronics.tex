Считывающая электроника жидкоаргоновых калориметров детектора ATLAS имеет сложную структуру, но в самом верхнем уровне её можно разделить на 2 части: фронтенд и бэкенд (или, как её ещё называют, задетекторная электроника). На рис. \ref{fig:read_electronics} изображена общая схема устройства считывающей электроники системы жидкоаргоновых калориметров.
\begin{figure}[ht]
    \centering
    \includegraphics[width=\linewidth]{read_electronics.png}
    \caption{Схема считывающей электроники жидкоаргоновых калориметров ATLAS}
    \label{fig:read_electronics}
\end{figure}\par
Фронтенд часть располагается в непосредственной близости с ускорителем, поэтому к ней предъявляются определённые требования по радиационной стойкости и отказоустойчивости. В рамках второй фазы обновления электроники на детектор будут установлены новые платы считывания FEB2 (FEB -- Front-End Board), а также платы калибровки.\par
Задетекторная часть удалена от радиационной зоны и принимает оцифрованные данные с фронтенда через оптические каналы связи. Именно в этой части выполняется цифровая фильтрация сигналов по каждой ячейке калориметра, их буферизация до появления сигнала триггерной системы и передача соответствующих данных в систему сбора данных DAQ (Data Acquisition).\par

\subsubsection{Модуль FEB2}
Платы FEB2 принимают сигналы от калориметрических ячеек и выполняют их аналоговую обработку, включая усиление, формирование и разделение на две перекрывающиеся шкалы линейного усиления. Обе шкалы усиления оцифровываются при помощи 14-битного АЦП, после чего цифровые синалы сериализуются и отправляются через оптический канал связи. Для этого используется несколько специализированных интегральных микросхем, а также системы управления и синхронизации. Каждая плата FEB2 способна обрабатывать 128 калориметрических каналов, а для считывания всей системы жидкоаргоновых калориметров требуется 1524 таких устройств.\par


\subsubsection{Система подготовки данных для триггера}
В целях получения как можно более быстрого решения триггерной системы, пусть даже и менее точного, в считывающей электронике жидкоаргоновых калориметров ATLAS предусмотрена система подготовки и передачи энергетических сумм по частям детектора в триггер. Такие сигналы генерируются в модуле FEB2, после чего в аналоговом виде отправляются на плату оцифровки триггера LTDB. Каждая такая плата способна обрабатывать до 320 сигналов, оцифровывая их с помощью 80 12-битных четырёхканальных АЦП \parencite{ltdb}. Далее эти значения передаются на двадцать трансиверов lpGBT, которые формируют 40 выходных потоков с объёмом данных 5,12 Гбит/с каждый для их отправки по волоконно-оптическим каналом связи в систему LDPS (LAr Digital Processing System). Всего в системе считывающей электроники предусмотрено 124 модуля LTDB, которые, соответственно, суммарно генерируют поток данных со скоростью примерно 25 Тбит/с.\par
Управление и мониторинг системы оцифровки данных триггера осуществляются по каналам связи 5 Гбит/с, подключенным через интерфейс обмена данными между фронтенд подключениями FELIX (Front-End LInks eXchange \parencite{felix}) в систему сбора данных и триггера ATLAS TDAQ (Trigger and Data Acquisition \parencite{tdaq}).\par
Обработанные в LTDB данные затем передаются в систему цифровой обработки LDPS, которая преобразует измерения АЦП в откалиброванные значения энергии в режиме реального времени. Система построена с использованием мезонинных плат расширений AMC (Advanced Mezzanine Card), которые выполняют точное восстановление энергии и определение настоящего времени столкновения пучков. Для реализации данных функций в платах расширения применяются программируемые логические интегральные схемы Altera Arria-10.\par


\subsubsection{Калибровочная система}
Важной частью фронтенд электроники является калибровочная система. С помощью специальных плат реализуется подача точных калибровочных сигналов непосредственно на ячейки жидкоаргонового калориметра. Форма калибровочного сингала максимально приближена к импульсу ионизации, генерируемому электромагнитным ливнем в детекторе. В силу того, что получить истинно треугольный сигнал с помощью электронной схемы достаточно трудно, первоначально создаётся экспоненциальный импульс, у которого обрезается область затухания для максимального приближения к желаемой треугольной форме, по крайней мере, в начальной части импульса. Для компенсации остаточной разницы в форме между физическим импульсом ионизации и калибровочным сигналом производится непосредственное измерение свойств последнего для их учёта в процедуре калибровки.\par


