В целях получения как можно более быстрого решения триггерной системы, пусть даже и менее точного, в считывающей электронике жидкоаргоновых калориметров ATLAS предусмотрена система подготовки и передачи энергетических сумм по частям детектора в триггер. Такие сигналы генерируются в модуле FEB2, после чего в аналоговом виде отправляются на плату оцифровки триггера LTDB. Каждая такая плата способна обрабатывать до 320 сигналов, оцифровывая их с помощью 80 12-битных четырёхканальных АЦП\parencite{ltdb}. Далее эти значения передаются на двадцать сериализаторов, реализованных в виде интегральных схем специального назначения, которые формируют 40 выходных потоков с объёмом данных 5,12 Гбит/с каждый для их отправки по волоконно-оптическим каналом связи в систему LDPS (LAr Digital Processing System). Всего в системе считывающей электроники предусмотренно 124 модуля LTDB, которые, соответственно, суммарно генерируют поток данных со скоростью примерно 25 Тбит/с.\par
Управление и мониторинг системы оцифровки данных триггера осуществляется по каналам связи 5 Гбит/с, подключенным через интерфейс обмена данными между фронтенд подключениями FELIX (Front-End LInks eXchange) в систему сбора данных и триггера ATLAS TDAQ (Trigger and Data Acquisition).\par
Обработанные в LTDB данные затем передаются в систему цифровой обработки LDPS, которая преобразует измерения АЦП в откалиброванные значения энергии в режиме реального времени. Система построена с использованием мезонинных плат расширений AMC (Advansed Mezzanine Card), которые выполняют точное восстановление энергии и определение настоящего времени столкновения пучков. Для реализации данных функций в платах расширения применяются программируемые логические интегральные схемы (ПЛИС) Altera Arria-10.\par
