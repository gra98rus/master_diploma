Важной частью фронтенд электроники является калибровочная система. С помощью специальных плат реализуется подача точных калибровочных сигналов непосредственно на ячейки жидкоаргонового калориметра. Форма калибровочного сигнала максимально приближена к импульсу ионизации, генерируемому электромагнитным ливнем в детекторе. В силу того, что получить истинно треугольный сигнал с помощью электронной схемы достаточно трудно, первоначально создаётся экспоненциальный импульс, у которого обрезается область затухания для максимального приближения к желаемой треугольной форме, по крайней мере, в начальной части импульса. Для компенсации остаточной разницы в форме между физическим импульсом ионизации и калибровочным сигналом производится непосредственное измерение свойств последнего для их учёта в процедуре калибровки.\par
