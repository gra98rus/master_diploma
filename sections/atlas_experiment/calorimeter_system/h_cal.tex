Торцевой адронный калориметр(HEC -- hadronic end-cap) детектора ATLAS состоит из двух независимых колёс, которые установлены за блоками торцевого электромагнитного калориметра. Он обеспечивает адронное покрытие псевдобыстроты в диапазоне $1,5 < |\eta| < 3,2$. По принципу действия торцевой адронный калориметр похож на электромагнитный, но имеет плоскопараллельную структуру внутренней геометрии с медными пластинами-поглотителями, а в качестве адсорбера в нём используется железо.\par
Оба колеса калориметра состоят из 32 одинаковых по азимуту модулей. Переднее колесо разделено по глубине на две секции считывания, которые суммарно содержат 24 слоя поглотителя. Заднее колесо выполнено из 16 слоёв поглотителя, объединённых в один сегмент считывания. С каждой полученной ячейки регистрируется отдельный сигнал. Для обеспечения наилучшего отношения сигнала и шума предусилители считывающей электроники калориметра находятся в среде с низкой температурой и расположены по внешнему радиусу модулей.\par
Важным аспектом адронного калориметра является его способность обнаруживать мюоны, а также измерять их любые ионизационные потери и треки.\par
