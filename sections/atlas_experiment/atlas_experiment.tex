Детектор ATLAS является детектором общего назначения, предназначенном для изучения протон-протонных столкновений, а также столкновений тяжелых ионов. С помощью ATLAS проводится широкий спектр исследований в области физики элементарных частиц; от поиска бозона Хиггса до попыток обнаружения частиц, которые могут составлять тёмную материю. Кроме того, одним из важнейших направлений является поиск новых физических явлений, которые не описываются стандартной моделью.\par
ATLAS охватывает приблизительно 99\% всего телесного угла вокруг точки столкновения и состоит из множества подсистем, в том числе жидкоаргоновых калориметров. Система калориметров обеспечивает измерение энергии частиц в широком диапазоне полярного угла детектора.\par

\subsection{Система жидкоаргоновых калориметров}
Система жидкоаргоновых калориметров детектора ATLAS имеет ключевую роль в измерении энергии и положения электронов, фотонов и заряженных адронов. Она состоит из четырёх основных частей \parencite{tdr_green} (рис. \ref{fig:atlas_cal}):
\begin {itemize}
    \item электромагнитная цилиндрическая;
    \item электромагнитная торцевая;
    \item адронная торцевая;
    \item форвард калориметр.
\end{itemize}\par
\begin{figure}[ht]
    \centering
    \includegraphics[width=0.8\linewidth]{atlas_cal.png}
    \caption{Схема системы жидкоаргоновых калориметров ATLAS}
    \label{fig:atlas_cal}
\end{figure}
Важной характеристикой системы калориметров является диапазон покрытия псевдобыстроты $|\eta|$. Эта величина показывает, насколько направление движения элементарной частицы отличается от оси пучка, и определяется как:
\begin{equation}
    \eta = -\ln(\tan(\frac{\Theta}{2})),
\end{equation}\par
где $\Theta$ -- угол между направлением импульса частицы и осью пучка. В физике коллайдеров зачастую используют именно этот показатель вместо простого полярного угла $\Theta$, так как плотность числа рождённых частиц приблизительно постоянна в единицу $|\eta|$. По этой причине калориметры обычно сегментируют по псевдобыстроте, а не по телесному углу. Калориметрическая система ATLAS охватывает диапазон $|\eta|$ до 4.9.

\subsubsection{Электромагнитный калориметр}
Для прецизионного детектирования и измерения электронов и фотонов калориметрическая система ATLAS включает в себя электромагнитный калориметр. Он состоит из центрального (баррельного) блока (EMB -- electromagnetic barrel), покрывающего диапазон псевдобыстрот $|\eta| < 1,475$, и пары торцевых частей (EMEC -- electromagnetic end-cap), соответствующих области $1,375 < |\eta| < 3,2$. Электромагнитные калориметры ATLAS построены по гетерогенному принципу, то есть в них разделены функции поглощения и детектирования. В качестве активного вещества служит жидкий аргон, находящийся при температуре около 90K, а для поглощающего материала используется свинец. Между пластинами поглотителя также располагаются медно-каптоновые электроды, по которым происходит снятие сигнала.\par
Заряженная частица, попадая в калориметр, порождает в нём электромагнитный ливень (рис. \ref{fig:em_shower})\parencite{em_shower_wiki}, который детектируется по принципу ионизационной камеры: под воздействием электрического поля между заземлённым поглотителем и электродом, находящимся под высоким напряжением, ионы и электроны дрейфуют, причём последние индуцируют треугольный импульс на электроде (рис. \ref{fig:tri_impulse}) (в действительности, сигнал является более сложным, чем просто треугольник -- в силу поглощения электронов загрязняющими примесями в активном веществе, такими как кислород или хлор, результирующий сигнал падает, а его форма домножается на небольшую экспоненту). Высота индуцированного импульса пропорциональна энергии, накопленной в ячейке калориметра. Время пика импульса используется для определения времени появления частицы.\par
\begin{figure}[ht]
    \centering
    \includegraphics[width=0.5\linewidth]{em_shower.png}
    \caption{Схема электромагнитного ливня}
    \label{fig:em_shower}
\end{figure}
\begin{figure}[ht]
    \centering
    \includegraphics[width=0.5\linewidth]{tri_impulse.png}
    \caption{Форма импульса тока электромагнитного калориметра и выходного сигнала после формирования}
    \label{fig:tri_impulse}
\end{figure}
Электромагнитный калориметр имеет сложную геометрию в форме гармошки (аккордеон). Это позволяет достичь полной симметрии калориметра по азимутальному углу, а также обеспечить высокую гранулированность детектора и увеличить его быстродействие за счёт малого зазора между пластинами. Толщина EMB составляет более 24 радиационных длин ($X_0$, расстояние, на котором интенсивность потока электронов высокой энергии и гамма-излучения падает в e раз). Каждый модуль калориметра имеет ячеистую структуру и поделён на несколько слоёв по глубине, как, например, модуль центрального блока на рис. \ref{fig:em_cal_struct}. Калориметр сконструирован так, что наибольшая часть энергии собирается в среднем слое, задний слой собирает лишь хвост электромагнитного потока. Передний слой сегментирован таким образом, чтобы с его помощью можно было максимально точно определить направление падающих частиц. Исходя из этого, используя измерение энергии и положения всех ячеек в каждом слое калориметра можно восстановить энергию и траекторию рождённых частиц.
\begin{figure}[ht]
    \centering
    \includegraphics[width=0.7\linewidth]{em_cal_struct.png}
    \caption{Схема разделения модуля EMB по слоям}
    \label{fig:em_cal_struct}
\end{figure}


\subsubsection{Торцевой адронный калориметр}
Торцевой адронный калориметр(HEC -- hadronic end-cap) детектора ATLAS состоит из двух независимых колёс, которые установлены за блоками торцевого электромагнитного калориметра. Он обеспечивает адронное покрытие псевдобыстроты в диапазоне $1,5 < |\eta| < 3,2$. По принципу действия торцевой адронный калориметр похож на электромагнитный, но имеет плоскопараллельную структуру внутренней геометрии с медными пластинами-поглотителями, а в качестве адсорбера в нём используется железо.\par
Оба колеса калориметра состоят из 32 одинаковых по азимуту модулей. Переднее колесо разделено по глубине на две секции считывания, которые суммарно содержат 24 слоя поглотителя. Заднее колесо выполнено из 16 слоёв поглотителя, объединённых в один сегмент считывания. С каждой полученной ячейки регистрируется отдельный сигнал. Для обеспечения наилучшего отношения сигнала и шума предусилители считывающей электроники калориметра находятся в среде с низкой температурой и расположены по внешнему радиусу модулей.\par
Важным аспектом адронного калориметра является его способность обнаруживать мюоны, а также измерять их любые ионизационные потери и треки.\par


\subsubsection{Форвард калориметр}
Форвард калориметр находится ближе всего к пучку и обеспечивает электромагнитную и адронную калориметрию в диапазоне $3,2 < |\eta| < 4.9$. Из-за своего расположения он подвергается очень сильному воздействию дозы облучения мощностью до $10^6$ ${^\text{Гр}} / _\text{год}$ и потока нейтронов с кинетической энергией более 100 кэВ до 109 $\text{см}^{-2}\text{c}^{-1}$\parencite{tdr_old}. С учётом этих условий форвард калориметр разрабатывался с использованием следующих принципов:
\begin{itemize}
    \item механическая простота из небольшого набора материалов;
    \item использование радстойких материалов;
    \item использование материалов с высоким значением Z;
    \item достижение максимальной проективной толщины(вдоль проективных лучей от точки столкновения частиц);
    \item достижение максимальной средней плотности.
\end{itemize}\par
Калориметр состоит из трёх модулей: электромагнитного и двух адронных. В электромагнитной секции в качестве материала адсорбера используется медь, тогда как в адронных -- вольфрам. Номинальные внешние размеры у всех трёх модулей равные. Внутренняя структура представляет собой матрицу шестигранных трубок, расположенных вдоль пучка, изготовленных из материала поглотителя, в которые концентрично установлены медные электроды(рис. \ref{fig:f_cal_struct}). Пространство между стенками трубок и электродами заполнено жидким аргоном, выполняющим роль активного вещества. Конструкция позволяет точно контролировать зазор между электродами.
\begin{figure}[ht]
    \centering
    \includegraphics[width=0.6\linewidth]{f_cal_struct.png}
    \caption{Схема внутренней структуры форвард калориметра\parencite{tdr_old}}
    \label{fig:f_cal_struct}
\end{figure}\par
В итоге, форвард калориметр способен работать в очень радиационно нагруженных условиях, но при этом имеет сравнительно низкое разрешение. Однако, учитывая тот факт, что проходящие через него частицы имеют одни из наибольших абсолютных энергий, относительная точность остаётся достаточно высокой и такого разрешения вполне хватает для решения существующих физических задач.


\subsection{Считывающая электроника}
Считывающая электроника жидкоаргоновых калориметров детектора ATLAS имеет сложную структуру, но в самом верхнем уровне её можно разделить на 2 части: фронтенд и бэкенд (или, как её ещё называют, задетекторная электроника). На рис. \ref{fig:read_electronics} изображена общая схема устройства считывающей электроники системы жидкоаргоновых калориметров.
\begin{figure}[ht]
    \centering
    \includegraphics[width=\linewidth]{read_electronics.png}
    \caption{Схема считывающей электроники жидкоаргоновых калориметров ATLAS}
    \label{fig:read_electronics}
\end{figure}\par
Фронтенд часть располагается в непосредственной близости с ускорителем, поэтому к ней предъявляются определённые требования по радиационной стойкости и отказоустойчивости. В рамках второй фазы обновления электроники на детектор будут установлены новые платы считывания FEB2 (FEB -- Front-End Board), а также платы калибровки.\par
Задетекторная часть удалена от радиационной зоны и принимает оцифрованные данные с фронтенда через оптические каналы связи. Именно в этой части выполняется цифровая фильтрация сигналов по каждой ячейке калориметра, их буферизация до появления сигнала триггерной системы и передача соответствующих данных в систему сбора данных DAQ (Data Acquisition).\par

\subsubsection{Модуль FEB2}
Платы FEB2 принимают сигналы от калориметрических ячеек и выполняют их аналоговую обработку, включая усиление, формирование и разделение на две перекрывающиеся шкалы линейного усиления. Обе шкалы усиления оцифровываются при помощи 14-битного АЦП, после чего цифровые синалы сериализуются и отправляются через оптический канал связи. Для этого используется несколько специализированных интегральных микросхем, а также системы управления и синхронизации. Каждая плата FEB2 способна обрабатывать 128 калориметрических каналов, а для считывания всей системы жидкоаргоновых калориметров требуется 1524 таких устройств.\par


\subsubsection{Система подготовки данных для триггера}
В целях получения как можно более быстрого решения триггерной системы, пусть даже и менее точного, в считывающей электронике жидкоаргоновых калориметров ATLAS предусмотрена система подготовки и передачи энергетических сумм по частям детектора в триггер. Такие сигналы генерируются в модуле FEB2, после чего в аналоговом виде отправляются на плату оцифровки триггера LTDB. Каждая такая плата способна обрабатывать до 320 сигналов, оцифровывая их с помощью 80 12-битных четырёхканальных АЦП \parencite{ltdb}. Далее эти значения передаются на двадцать трансиверов lpGBT, которые формируют 40 выходных потоков с объёмом данных 5,12 Гбит/с каждый для их отправки по волоконно-оптическим каналом связи в систему LDPS (LAr Digital Processing System). Всего в системе считывающей электроники предусмотрено 124 модуля LTDB, которые, соответственно, суммарно генерируют поток данных со скоростью примерно 25 Тбит/с.\par
Управление и мониторинг системы оцифровки данных триггера осуществляются по каналам связи 5 Гбит/с, подключенным через интерфейс обмена данными между фронтенд подключениями FELIX (Front-End LInks eXchange \parencite{felix}) в систему сбора данных и триггера ATLAS TDAQ (Trigger and Data Acquisition \parencite{tdaq}).\par
Обработанные в LTDB данные затем передаются в систему цифровой обработки LDPS, которая преобразует измерения АЦП в откалиброванные значения энергии в режиме реального времени. Система построена с использованием мезонинных плат расширений AMC (Advanced Mezzanine Card), которые выполняют точное восстановление энергии и определение настоящего времени столкновения пучков. Для реализации данных функций в платах расширения применяются программируемые логические интегральные схемы Altera Arria-10.\par


\subsubsection{Калибровочная система}
Важной частью фронтенд электроники является калибровочная система. С помощью специальных плат реализуется подача точных калибровочных сигналов непосредственно на ячейки жидкоаргонового калориметра. Форма калибровочного сингала максимально приближена к импульсу ионизации, генерируемому электромагнитным ливнем в детекторе. В силу того, что получить истинно треугольный сигнал с помощью электронной схемы достаточно трудно, первоначально создаётся экспоненциальный импульс, у которого обрезается область затухания для максимального приближения к желаемой треугольной форме, по крайней мере, в начальной части импульса. Для компенсации остаточной разницы в форме между физическим импульсом ионизации и калибровочным сигналом производится непосредственное измерение свойств последнего для их учёта в процедуре калибровки.\par



