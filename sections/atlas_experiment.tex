Эксперимент ATLAS является одним из четырёх основных экспериментов на Большом адронном коллайдепре(БАК). Он проводится на одноимённом детекторе общего назначения, предназначенном для изучения протон-протонных столкновений, а также столкновений тяжелых ионов. С помощью детектора ATLAS проводится широкий спектр исследований в области физики элементарных частиц от поиска бозона Хиггса, до попыток обнаружения частиц, которые могут составлять тёмную материю. Кроме того, одним из важных направлений является поиск новых физических явлений, которые не описываются стандартной моделью.
