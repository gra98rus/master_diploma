Главной целью данной работы является разработка блока упаковки данных для системы FEX модуля сигнального процессора LASP жидкоаргонового калориметра детектора ATLAS. Эта подсистема состоит из пары связки компонентов, а именно упаковщика данных packer для системы fFEX и  модуля конфигурируемой перестановки remap. То есть по каждому из компонентов необходимо проработать их внутреннюю архитектуру, после чего реализовать на языке описания цифровой логики, что также подразумевает под собой:\par
\begin{itemize}
    \item написание синтезируемых блоков логической аппаратуры;
    \item создание симуляционного окружения и отладка разработанной структуры;
    \item компиляция модулей под целевую платформу;
    \item проверка и оптимизация занимаемых логических ресурсов и временных задержек.
\end{itemize}\par
Помимо работы по непосредственной реализации указанных компонентов сигнального процессора LASP необходимо разработать:\par
\begin{itemize}
    \item программное обеспечение для автоматической генерации конфигураций модуля remap;
    \item формат кадра протокола передачи данных из модуля packer сигнального процессора LASP в систему fFEX.
\end{itemize}\par
В рамках проекта LASP коллаборации ATLAS принято использовать для работы язык описания аппаратуры VHDL. В качестве наиболее возможной потенциальной микросхемы ПЛИС на данный момент рассматривается кристаллы от компании Intel, принадлежащие высокопроизводительному семейству Stratix, а именно Intel Stratix 10 SX 1SX280HU1F50E2VG, являющийся системой на кристалле, имеющей наряду с программируемой логикой также производительный процессор ARM A53 или Intel Stratix 10 MX 1SM21BHU1F53E2VG. Поскольку целевым устройством в любом случае является продукция Intel, то, соответственно, в качестве инструмента разработки ключевую роль занимает программное обеспечение Intel Quartus Prime. Также, в рамках данного проекта применяется симулятор цифровых логических схем Mentor QuestaSim.\par
