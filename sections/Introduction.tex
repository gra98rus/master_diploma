ATLAS -- это один из четырёх основных экспериментов на Большом Адронном коллайдере. Эксперимент проводится на одноимённом детекторе, предназначенном для исследования протон-протонных столкновений и столкновений тяжелых ионов. Детектор охватывает приблизительно 99\% всего телесного угла вокруг точки столкновения и состоит из множества подсистем, в том числе жидкоаргоновых калориметров. Система калориметров обеспечивает измерение энергии частиц в широком диапазоне полярного угла детектора.\par
В рамках второй фазы обновления системы жидкоаргоновых детекторов ATLAS будет произведена полная замена считывающей электроники из-за ограниченной радиационной стойкости некоторых установленных ранее компонентов, а также несовместимости с модернизированной триггерной системой. Важным компонентом новой считывающей системы будет являться сигнальный процессор LASP(Liquid Argon Signal Processor), реализующий приём оцифрованных сигналов, цифровую фильтрацию и буферизацию данных до момента принятия решения триггерной системы. В основе данного модуля будет работать микросхема программируемой логики(ПЛИС). В настоящее время ведётся активная разработка этого процессора, частью которой является данная работа.\par
