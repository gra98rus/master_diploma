ATLAS -- это один из четырёх основных экспериментов на Большом Адронном коллайдере (БАК). Эксперимент проводится на одноимённом детекторе, предназначенном для исследования протон-протонных столкновений и столкновений тяжелых ионов. Экспериментальные данные, полученные на многоцелевом детекторе ATLAS, используются для дальнейшего изучения свойств бозона Хиггса, поиска суперсимметричных частиц и широкого набора других задач.\par

В рамках второй фазы обновления системы жидкоаргоновых детекторов ATLAS ведётся проектирование совершенно новой системы считывающей электроники, которая будет установлена в период третьего длительного отключения БАК (2024 -- 2026 гг.). Это позволит расширить возможности эксперимента после модернизации Большого Адронного коллайдера, в результате которой ожидается значительное повышение мгновенной светимости до $7,5*10^{34} \text{см}^{-2}\text{c}^{-1}$ с целью обеспечения интегральной светимости $4000 \text{фб}^{-1}$ через период около 12 лет. Это позволит использовать БАК для исследования "новой физики", лежащей за границами Стандартной Модели.\par

Важным компонентом новой считывающей системы будет являться сигнальный процессор LASP (Liquid Argon Signal Processor), реализующий приём оцифрованных сигналов, цифровую фильтрацию и буферизацию данных до момента принятия решения триггерной системы. В основе данного модуля будет работать микросхема программируемой логики (ПЛИС). В настоящее время ведётся активная разработка этого процессора, частью которой является данная работа.\par
