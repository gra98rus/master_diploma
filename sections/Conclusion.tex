В рамках данной работы велась разработка блока упаковки данных сигнального процессора жидкоаргоновых калориметров LASP для системы FEX, состоящей из связки модулей конфигурируемой перестановки \texttt{remap} и упаковщика триггерных данных \texttt{packer}. Таким образом, были реализованы следующие задачи:\par
\begin{itemize}
    \item по модулю конфигурируемой перестановки \texttt{remap}:
        \begin{itemize}
            \item проработана внутренняя архитектура -- составлено 2 альтернативных варианта;
            \item написаны синтезируемые блоки цифровой логики на языке VHDL;
            \item создано симуляционное окружение для моделирования поведения модуля;
            \item проведена компиляция под целевую платформу;
            \item выполнена интеграция модуля в основную структуру сигнального процессора LASP;
            \item разработан прототип программного обеспечения для автоматической генерации конфигурации;
        \end{itemize}
    \item по упаковщику триггерных данных \texttt{packer} для системы fFEX:
        \begin{itemize}
            \item разработаны протоколы упаковки данных в кадры;
            \item реализован синтезируемый прототип модуля;
            \item создано симуляционное окружение для моделирования поведения модуля;
            \item выполнена интеграция модуля в основную структуру сигнального процессора LASP.
        \end{itemize}
\end{itemize}\par
В дальнейшем планируется масштабирование программного обеспечения для автоматической генерации конфигураций модуля \texttt{remap} на все участки жидкоаргонового калориметра. После согласования требований с разработчиками системы fFEX будет завершена реализация упаковщика \texttt{packer}. После завершения разработки предстоит запуск систем на целевой платформе и последующий ввод в эксплуатацию.\par
