\documentclass[a4paper, 14pt]{extarticle}
%\usepackage[utf8]{inputenc}
%\usepackage[english,russian]{babel}
%\usepackage{indentfirst}

%\usepackage{graphicx}
%\usepackage{float}
%\usepackage{wrapfig}
%\graphicspath{{images/}}

%\title{Электроника стенда по изучению сцинтилляционных кристаллов}
%\author{Андреев Андрей\\Новосибирский Государствнный Университет}

\input{sections/preamble.tex}
\begin{document}
%\maketitle
%\newpage
\newcommand{\fnt}[1]{\fontsize{#1}{\baselineskip}\selectfont}
% Ставим геометрию для титульной страницы.
\newgeometry{left=3cm,right=1.5cm,top=2cm,bottom=2cm}

% Просто информация. Нигде в странице не используется, но зато попадёт в pdf info.
\author{Мингулов К.\:Т.}
\title{Пакет DREAM: автоматизация метода многопетлевых вычислений, основанного на рекуррентных соотношениях по размерности пространства-времени}
\date{2022 г.}
\begin{titlepage}
\begin{spacing}{1}
%%%%%%%%%%%%%%%%%%%%%%%%%%%%%%%%%%%%%%%%%%%%%%%%%%%%%%%%%%%%%%%%%%%%%%%%%%%%%%%%

\begin{center}
    {\fnt{10.5} МИНИСТЕРСТВО НАУКИ И ВЫСШЕГО ОБРАЗОВАНИЯ РОССИЙСКОЙ ФЕДЕРАЦИИ} \\
        \vspace{0.3\baselineskip}
    {\fnt{10.5} ФЕДЕРАЛЬНОЕ ГОСУДАРСТВЕННОЕ АВТОНОМНОЕ ОБРАЗОВАТЕЛЬНОЕ \\
		\vspace{-0.3\baselineskip}
		УЧРЕЖДЕНИЕ ВЫСШЕГО ОБРАЗОВАНИЯ} \\
        \vspace{0.3\baselineskip}
    {\fnt{10.5} <<\mbox{\textls[-18]{НОВОСИБИРСКИЙ НАЦИОНАЛЬНЫЙ ИССЛЕДОВАТЕЛЬСКИЙ ГОСУДАРСТВЕННЫЙ}}\\
		\vspace{-0.3\baselineskip}
		УНИВЕРСИТЕТ>> (НОВОСИБИРСКИЙ ГОСУДАРСТВЕННЫЙ УНИВЕРСИТЕТ, НГУ)}
\end{center}

%%%%%%%%%%%%%%%%%%%%%%%%%%%%%%%%%%%%%%%%%%%%%%%%%%%%%%%%%%%%%%%%%%%%%%%%%%%%%%%%

\vspace{0.5\baselineskip}

\noindent
{\fnt{11}Факультет}
$\overset{\text{\fnt{11}\textbf{ФИЗИЧЕСКИЙ}}\phantom{\hspace{10.9cm}}}{\underline{\hspace{0.88\textwidth}}}$

\vspace{0.3\baselineskip}

\noindent
{\fnt{11}Кафедра}
$\overset{\text{\fnt{11}\textbf{ФИЗИКО-ТЕХНИЧЕСКОЙ ИНФОРМАТИКИ}}\phantom{\hspace{4.8cm}}}{\underline{\hspace{0.895\textwidth}}}$

\vspace{1\baselineskip}

\noindent
{\fnt{11}Направление подготовки}
$\overset{\text{\fnt{11}\textbf{03.03.02 ФИЗИКА}}\phantom{\hspace{8cm}}}{\underline{\hspace{0.73\textwidth}}}$

\vspace{0.3\baselineskip}

\noindent
{\fnt{11}Образовательная программа}
$\overset{\text{\fnt{11}\textbf{МАГИСТРАТУРА}}\phantom{\hspace{7.3cm}}}{\underline{\hspace{0.689\textwidth}}}$

%%%%%%%%%%%%%%%%%%%%%%%%%%%%%%%%%%%%%%%%%%%%%%%%%%%%%%%%%%%%%%%%%%%%%%%%%%%%%%%%

\vspace{\baselineskip}

\begin{center}\bfseries
    {\fnt{11} ВЫПУСКНАЯ КВАЛИФИКАЦИОННАЯ РАБОТА} \\
        \vspace{0\baselineskip}
    %{\fnt{11} МАГИСТЕРСКАЯ ДИССЕРТАЦИЯ}
\end{center}

%%%%%%%%%%%%%%%%%%%%%%%%%%%%%%%%%%%%%%%%%%%%%%%%%%%%%%%%%%%%%%%%%%%%%%%%%%%%%%%%

\vspace{0.3\baselineskip}

\noindent
$\overset{\text{\fnt{11}Андреев Андрей Андреевич}}
{\underset{\text{\fnt{9}(Фамилия, Имя, Отчество автора)}}{\underline{\hspace{\textwidth}}}}$

%%%%%%%%%%%%%%%%%%%%%%%%%%%%%%%%%%%%%%%%%%%%%%%%%%%%%%%%%%%%%%%%%%%%%%%%%%%%%%%%

\vspace{\baselineskip}

\noindent
{\fnt{11}Тема работы}
$\overset{\text{\fnt{11}Блок упаковки данных для системы FEX модуля LASP жидкоаргонового }}{\underline{\hspace{0.855\textwidth}}}$

\noindent
$\overset{\text{\fnt{11}калориметра детектора ATLAS}}{\underline{\hspace{\textwidth}}}$

\noindent
$\overset{\text{\fnt{11}}}{\underline{\hspace{\textwidth}}}$

%%%%%%%%%%%%%%%%%%%%%%%%%%%%%%%%%%%%%%%%%%%%%%%%%%%%%%%%%%%%%%%%%%%%%%%%%%%%%%%%

\vspace{2\baselineskip}

\noindent
\begin{tabular}{@{}p{0.5\textwidth}@{}@{}R{0.5\textwidth}@{}}
\fnt{11}\textbf{<<К защите допущена>>} &  \\
\fnt{11}Декан ФФ НГУ                   & \fnt{11}\textbf{Научный руководитель} \\
\fnt{11}д.ф.-м.н.           & \fnt{11}канд. техн. наук \\
\fnt{11}зав. лаб. 3-2 ИЯФ СО РАН               & \fnt{11}зав. сек. 3-12 ИЯФ СО РАН \\
$\overset{\text{\fnt{11}Блинов В.\,Е.}}{\underset{\text{\fnt{8}(фамилия, И., О.)}}{\underline{\hspace{0.225\textwidth}}}}
\overset{\text{\fnt{11}/}}{}
\overset{\text{\fnt{11}}}{\underset{\text{\fnt{8}(подпись, МП)}}{\underline{\hspace{0.125\textwidth}}}}$ &
$\overset{\text{\fnt{11}Жуланов В.\,В.}}{\underset{\text{\fnt{8}(фамилия, И.,О.)}}{\underline{\hspace{0.225\textwidth}}}}
\overset{\text{\fnt{11}/}}{}
\overset{\text{\fnt{11}}}{\underset{\text{\fnt{8}(подпись, МП)}}{\underline{\hspace{0.125\textwidth}}}}$ \\
$\overset{\text{\fnt{11}<<}}{}
\overset{\text{\fnt{11}}}{\underline{\hspace{0.05\textwidth}}}
\overset{\text{\fnt{11}>>}}{}
\overset{\text{\fnt{11}}}{\underline{\hspace{0.215\textwidth}}}
\overset{\text{\fnt{11}2022 г.}}{}$ &
$\overset{\text{\fnt{11}<<}}{}
\overset{\text{\fnt{11}}}{\underline{\hspace{0.05\textwidth}}}
\overset{\text{\fnt{11}>>}}{}
\overset{\text{\fnt{11}}}{\underline{\hspace{0.215\textwidth}}}
\overset{\text{\fnt{11}2022 г.}}{}$
\end{tabular}

%%%%%%%%%%%%%%%%%%%%%%%%%%%%%%%%%%%%%%%%%%%%%%%%%%%%%%%%%%%%%%%%%%%%%%%%%%%%%%%%

\vspace{1.5\baselineskip}

\begin{flushright}
\fnt{11}
$\overset{\text{\fnt{11}Дата защиты: <<}}{}
\overset{\text{\fnt{11}}}{\underline{\hspace{0.05\textwidth}}}
\overset{\text{\fnt{11}>>}}{}
\overset{\text{\fnt{11}}}{\underline{\hspace{0.215\textwidth}}}
\overset{\text{\fnt{11}2022 г.}}{}$
\end{flushright}

%%%%%%%%%%%%%%%%%%%%%%%%%%%%%%%%%%%%%%%%%%%%%%%%%%%%%%%%%%%%%%%%%%%%%%%%%%%%%%%%

\vfill

\begin{center}
    \fnt{11} Новосибирск, 2022
\end{center}

\end{spacing}
\end{titlepage}

\restoregeometry


\tableofcontents
\thispagestyle{empty}
\newpage

\setcounter{page}{3}
\section*{Введение}
\addcontentsline{toc}{section}{Введение}
    ATLAS -- это один из четырёх основных экспериментов на Большом Адронном коллайдере (БАК). Эксперимент проводится на одноимённом детекторе, предназначенном для исследования протон-протонных столкновений и столкновений тяжелых ионов. Экспериментальные данные, полученные на многоцелевом детекторе ATLAS, используются для дальнейшего изучения свойств бозона Хиггса, поиска суперсимметричных частиц и широкого набора других задач.\par

В рамках второй фазы обновления системы жидкоаргоновых детекторов ATLAS ведётся проектирование совершенно новой системы считывающей электроники, которая будет установлена в период третьего длительного отключения БАК (2024 -- 2026 гг.). Это позволит расширить возможности эксперимента после модернизации Большого Адронного коллайдера, в результате которой ожидается значительное повышение мгновенной светимости до $7,5*10^{34} \text{см}^{-2}\text{c}^{-1}$ с целью обеспечения интегральной светимости $4000 \text{фб}^{-1}$ через период около 12 лет. Это позволит использовать БАК для исследования "новой физики", лежащей за границами Стандартной Модели.\par

Важным компонентом новой считывающей системы будет являться сигнальный процессор LASP (Liquid Argon Signal Processor), реализующий приём оцифрованных сигналов, цифровую фильтрацию и буферизацию данных до момента принятия решения триггерной системы. В основе данного модуля будет работать микросхема программируемой логики (ПЛИС). В настоящее время ведётся активная разработка этого процессора, частью которой является данная работа.\par

    \newpage

\section{Эксперимент ATLAS}
    Детектор ATLAS является одним из двух крупных экспериментов общего назначения, предназначенных для изучения протон-протонных столкновений, а также столкновений тяжелых ионов на Большом адронном коллайдере(БАК).

    \subsection{Калориметрическая система}
    Система жидкоаргоновых калориметров детектора ATLAS имеет ключевую роль в измерении энергии и положения электронов, фотонов и заряженных адронов. Она состоит из четырёх основных частей \parencite{tdr_green} (рис. \ref{fig:atlas_cal}):
\begin {itemize}
    \item электромагнитная цилиндрическая;
    \item электромагнитная торцевая;
    \item адронная торцевая;
    \item форвард калориметр.
\end{itemize}\par
\begin{figure}[ht]
    \centering
    \includegraphics[width=0.8\linewidth]{atlas_cal.png}
    \caption{Схема системы жидкоаргоновых калориметров ATLAS}
    \label{fig:atlas_cal}
\end{figure}
Важной характеристикой системы калориметров является диапазон покрытия псевдобыстроты $|\eta|$. Эта величина показывает, насколько направление движения элементарной частицы отличается от оси пучка, и определяется как:
\begin{equation}
    \eta = -\ln(\tan(\frac{\Theta}{2})),
\end{equation}\par
где $\Theta$ -- угол между направлением импульса частицы и осью пучка. В физике коллайдеров зачастую используют именно этот показатель вместо простого полярного угла $\Theta$, так как плотность числа рождённых частиц приблизительно постоянна в единицу $|\eta|$. По этой причине калориметры обычно сегментируют по псевдобыстроте, а не по телесному углу. Калориметрическая система ATLAS охватывает диапазон $|\eta|$ до 4.9.

\subsubsection{Электромагнитный калориметр}
Для прецизионного детектирования и измерения электронов и фотонов калориметрическая система ATLAS включает в себя электромагнитный калориметр. Он состоит из центрального (баррельного) блока (EMB -- electromagnetic barrel), покрывающего диапазон псевдобыстрот $|\eta| < 1,475$, и пары торцевых частей (EMEC -- electromagnetic end-cap), соответствующих области $1,375 < |\eta| < 3,2$. Электромагнитные калориметры ATLAS построены по гетерогенному принципу, то есть в них разделены функции поглощения и детектирования. В качестве активного вещества служит жидкий аргон, находящийся при температуре около 90K, а для поглощающего материала используется свинец. Между пластинами поглотителя также располагаются медно-каптоновые электроды, по которым происходит снятие сигнала.\par
Заряженная частица, попадая в калориметр, порождает в нём электромагнитный ливень (рис. \ref{fig:em_shower})\parencite{em_shower_wiki}, который детектируется по принципу ионизационной камеры: под воздействием электрического поля между заземлённым поглотителем и электродом, находящимся под высоким напряжением, ионы и электроны дрейфуют, причём последние индуцируют треугольный импульс на электроде (рис. \ref{fig:tri_impulse}) (в действительности, сигнал является более сложным, чем просто треугольник -- в силу поглощения электронов загрязняющими примесями в активном веществе, такими как кислород или хлор, результирующий сигнал падает, а его форма домножается на небольшую экспоненту). Высота индуцированного импульса пропорциональна энергии, накопленной в ячейке калориметра. Время пика импульса используется для определения времени появления частицы.\par
\begin{figure}[ht]
    \centering
    \includegraphics[width=0.5\linewidth]{em_shower.png}
    \caption{Схема электромагнитного ливня}
    \label{fig:em_shower}
\end{figure}
\begin{figure}[ht]
    \centering
    \includegraphics[width=0.5\linewidth]{tri_impulse.png}
    \caption{Форма импульса тока электромагнитного калориметра и выходного сигнала после формирования}
    \label{fig:tri_impulse}
\end{figure}
Электромагнитный калориметр имеет сложную геометрию в форме гармошки (аккордеон). Это позволяет достичь полной симметрии калориметра по азимутальному углу, а также обеспечить высокую гранулированность детектора и увеличить его быстродействие за счёт малого зазора между пластинами. Толщина EMB составляет более 24 радиационных длин ($X_0$, расстояние, на котором интенсивность потока электронов высокой энергии и гамма-излучения падает в e раз). Каждый модуль калориметра имеет ячеистую структуру и поделён на несколько слоёв по глубине, как, например, модуль центрального блока на рис. \ref{fig:em_cal_struct}. Калориметр сконструирован так, что наибольшая часть энергии собирается в среднем слое, задний слой собирает лишь хвост электромагнитного потока. Передний слой сегментирован таким образом, чтобы с его помощью можно было максимально точно определить направление падающих частиц. Исходя из этого, используя измерение энергии и положения всех ячеек в каждом слое калориметра можно восстановить энергию и траекторию рождённых частиц.
\begin{figure}[ht]
    \centering
    \includegraphics[width=0.7\linewidth]{em_cal_struct.png}
    \caption{Схема разделения модуля EMB по слоям}
    \label{fig:em_cal_struct}
\end{figure}


\subsubsection{Торцевой адронный калориметр}
Торцевой адронный калориметр(HEC -- hadronic end-cap) детектора ATLAS состоит из двух независимых колёс, которые установлены за блоками торцевого электромагнитного калориметра. Он обеспечивает адронное покрытие псевдобыстроты в диапазоне $1,5 < |\eta| < 3,2$. По принципу действия торцевой адронный калориметр похож на электромагнитный, но имеет плоскопараллельную структуру внутренней геометрии с медными пластинами-поглотителями, а в качестве адсорбера в нём используется железо.\par
Оба колеса калориметра состоят из 32 одинаковых по азимуту модулей. Переднее колесо разделено по глубине на две секции считывания, которые суммарно содержат 24 слоя поглотителя. Заднее колесо выполнено из 16 слоёв поглотителя, объединённых в один сегмент считывания. С каждой полученной ячейки регистрируется отдельный сигнал. Для обеспечения наилучшего отношения сигнала и шума предусилители считывающей электроники калориметра находятся в среде с низкой температурой и расположены по внешнему радиусу модулей.\par
Важным аспектом адронного калориметра является его способность обнаруживать мюоны, а также измерять их любые ионизационные потери и треки.\par


\subsubsection{Форвард калориметр}
Форвард калориметр находится ближе всего к пучку и обеспечивает электромагнитную и адронную калориметрию в диапазоне $3,2 < |\eta| < 4.9$. Из-за своего расположения он подвергается очень сильному воздействию дозы облучения мощностью до $10^6$ ${^\text{Гр}} / _\text{год}$ и потока нейтронов с кинетической энергией более 100 кэВ до 109 $\text{см}^{-2}\text{c}^{-1}$\parencite{tdr_old}. С учётом этих условий форвард калориметр разрабатывался с использованием следующих принципов:
\begin{itemize}
    \item механическая простота из небольшого набора материалов;
    \item использование радстойких материалов;
    \item использование материалов с высоким значением Z;
    \item достижение максимальной проективной толщины(вдоль проективных лучей от точки столкновения частиц);
    \item достижение максимальной средней плотности.
\end{itemize}\par
Калориметр состоит из трёх модулей: электромагнитного и двух адронных. В электромагнитной секции в качестве материала адсорбера используется медь, тогда как в адронных -- вольфрам. Номинальные внешние размеры у всех трёх модулей равные. Внутренняя структура представляет собой матрицу шестигранных трубок, расположенных вдоль пучка, изготовленных из материала поглотителя, в которые концентрично установлены медные электроды(рис. \ref{fig:f_cal_struct}). Пространство между стенками трубок и электродами заполнено жидким аргоном, выполняющим роль активного вещества. Конструкция позволяет точно контролировать зазор между электродами.
\begin{figure}[ht]
    \centering
    \includegraphics[width=0.6\linewidth]{f_cal_struct.png}
    \caption{Схема внутренней структуры форвард калориметра\parencite{tdr_old}}
    \label{fig:f_cal_struct}
\end{figure}\par
В итоге, форвард калориметр способен работать в очень радиационно нагруженных условиях, но при этом имеет сравнительно низкое разрешение. Однако, учитывая тот факт, что проходящие через него частицы имеют одни из наибольших абсолютных энергий, относительная точность остаётся достаточно высокой и такого разрешения вполне хватает для решения существующих физических задач.


    \subsection{Считывающая электроника}
    Считывающая электроника жидкаргоновых калориметров детектора ATLAS имеет сложную структуру, но в самом верхнем уровне её можно разделить на 2 части: фронтенд и задетекторную электронику. На рис. \ref{fig:read_electronics} изображена общая схема устройства считывающей электроники системы жидкоаргоновы калориметров.
\begin{figure}[ht]
    \centering
    \includegraphics[width=\linewidth]{read_electronics.png}
    \caption{Схема считывающей электроники жидкоаргоновых калориметров ATLAS}
    \label{fig:read_electronics}
\end{figure}\par
Фронтенд часть располагается в непосредственной близости с ускорителем, поэтому на неё налагаются определённые требования по радиационной стойкости и отказоустойчивости. В рамках второй фазы обновления электроники на детектор будут установлены новые платы считывания FEB2(FEB -- Front-End Board), а также платы калибровки.\par

\subsubsection{Модуль FEB2}
Платы FEB2 принимают сигналы от калориметрических ячеек и выполняют их аналоговую обработку, включая усиление, формирование и разделение на две перекрывающиеся шкалы линейного усиления. Обе шкалы усиления оцифровываются при помощи 14-битного АЦП, после чего цифровые синалы сериализуются и отправляются через оптический канал связи. Для этого используется несколько специализированных интегральных микросхем, а также системы управления и синхронизации. Каждая плата FEB2 способна обрабатывать 128 калориметрических каналов, а для считывания всей системы жидкоаргоновых калориметров требуется 1524 таких устройств.\par


\subsubsection{Модуль LTDB}
\input{subsections/ltdb.tex}

\subsubsection{Калибровочная система}
Важной частью фронтенд электроники является калибровочная система. С помощью специальных плат реализуется подача точных калибровочных сигналов непосредственно на ячейки жидкоаргонового калориметра. Форма калибровочного сингала максимально приближена к импульсу ионизации, генерируемому электромагнитным ливнем в детекторе. В силу того, что получить истинно треугольный сигнал с помощью электронной схемы достаточно трудно, первоначально создаётся экспоненциальный импульс, у которого обрезается область затухания для максимального приближения к желаемой треугольной форме, по крайней мере, в начальной части импульса. Для компенсации остаточной разницы в форме между физическим импульсом ионизации и калибровочным сигналом производится непосредственное измерение свойств последнего для их учёта в процедуре калибровки.\par



    \newpage

\section{Сигнальный процессор жидкоаргонового калориметра (LASP)}
    Основным элементом задетекторной считывающей электроники жидкоаргонового калориметра детектора ATLAS в рамках второй фазы обновления являются модули сигнального процессора LASP (Liquid Argon Signal Processor). Они предназначены для принятия оцифрованных данных с модулей FEB2 и применения к ним цифровой фильтрации, их буферизации до появления сигнала триггера и последующей передачи в систему сбора данных DAQ. Также система LASP обеспечивает подготовку входных данных для таких систем, как глобальный триггер и fFEX (forward Feature EXtractor). Система глобального триггера будет получать значения энергий только от тех ячеек, которые превышают заданный порог, определённый относительно общего шума. Таким образом, полосой пропускания данных можно управлять, сохраняя при этом достаточную количество информации для кластеризации событий.\par
Сигнальные процессоры рассчитаны на приём непрерывного потока оцифрованных данных с плат FEB2 на частоте соударения пучков частиц в Большом Адронном коллайдере (фактическая частота составляет 40.07897 МГц) для всех 182486 ячеек жидкоаргонового калориметра. Каждый модуль будет получать исходные данные с 8 плат FEB2, то есть с 1024 калориметрических ячеек. В настоящее время ведётся активная разработка этой системы.\par
Плата сигнального процессора LASP изготовлена в формате модуля ATCA (Advanced Telecom Computing Architecture \parencite{atca}). Модули LASP требуют высокой пропускной способности ввода и вывода, а также возможности гибкого программирования алгоритмов обработки данных, цифровой фильтрации и сокращения объёма данных, поэтому в качестве основных вычислительных блоков LASP предусмотрены программируемые интегральные микросхемы. На плате каждого модуля будет располагаться 2 таких чипа для увеличения пропускной способности. Внутренняя структура дизайна программируемой логики представлена на рисунке \ref{fig:lasp}.

\begin{figure}[ht]
    \centering
    \includegraphics[width=\linewidth]{lasp.png}
    \caption{Блок схема архитектуры сигнального процессора LASP}
    \label{fig:lasp}
\end{figure}\par

Основными модулями сигнального процессора LASP являются:\par
\begin{itemize}
    \item интерфейс нижнего уровня \texttt{lolli};
    \item система медленного контроля \texttt{sctrl};
    \item генератор тестовых сигналов \texttt{patgen};
    \item выравниватель входных данных \texttt{ialign};
    \item модуль конфигурируемой перестановки \texttt{remap};
    \item ядро обработки данных \texttt{dacore};
    \item процессор онлайн светимости \texttt{olump};
    \item упаковщик триггерных данных \texttt{packer};
    \item блок буферов \texttt{buffs};
    \item модуль форматирования данных \texttt{fbuild};
    \item модуль мониторинга данных \texttt{damon};
    \item монитор состояния аппаратуры \texttt{bomon}.
\end{itemize}\par

Для работы сигнального процессора используется целый набор различных тактовых сигналов. Среди основных можно выделить:
\begin{itemize}
    \item $f_{feb}$ -- тактовая частота, синхронно с которой поступают входные данные с системы FEB2. Имеет фиксированное значение 320 МГц;
    \item $f_{core}$ --  тактовая частота, синхронно с которой происходит непосредственная обработка данных. В зависимости от конфигурации может быть либо 320 МГц -- так называемая медленная опция, либо 480 МГц -- быстрая опция;
    \item $f_{sctrl}$ -- тактовая частота, на которой функционирует интерфейс медленного контроля. Непосредственное значение составляет 100 МГц.
    \item $f_{xgbe}$ -- тактовая частота, необходимая для приёма и отправки данных через 10 Гбит Ethernet порт(X Gigabit Ethernet). Является стандартной для такого порта и составляет 156,25 МГц.
\end{itemize}\par
Также в системе присутствует ещё несколько вспомогательных тактовых сигналов, необходимых для работы DDR4 интерфейса и TTC RX.\par
\textbf{Интерфейс нижнего уровня \texttt{lolli}}\par
Базовым модулем системы LASP, с помощью которого осуществляется взаимодействие с внешним миром, является интерфейс нижнего уровня \texttt{lolli}. Данная подсистема содержит реализации всех необходимых низкоуровневых внешних интерфейсов:\par
\begin{itemize}
    \item FEB2;
    \item Gigabit Ethernet;
    \item 10 Gigabit Ethernet;
    \item DDR4 SDRAM;
    \item I2C;
    \item Custom BUS;
    \item fFEX;
    \item FELIX;
    \item Global Trigger.
\end{itemize}\par
При возможности, все интерфейсы из \texttt{lolli} в программируемую логику спроектированы с использованием стандартных потокового интерфейса Avalon Stream (AVST) и интерфейса, отображаемого на память Avalon Memory Mapped (AVMM). Это позволяет иметь четко определённые и документированные стандартные интерфейсы между каждым компонентом LASP.\par
\textbf{Система медленного контроля \texttt{sctrl}}\par
Для реализации возможности управления всеми компонентами сигнального процессора, а также их соединения с внешним миром предусмотрена система медленного контроля \texttt{sctrl}. Она позволяет пользователю загружать или изменять все доступные пользователю параметры конфигурации, а также иметь доступ ко всем регистрам мониторинга и состояния любого модуля в режиме реального времени.\par
Компонент \texttt{sctrl} использует внешний канал связи Gigabit Ethernet, реализованный в интерфейсе \texttt{lolli}. Для общения с внутренними модулями используется AVMM интерфейс. Специально для интерфейса медленного контроля каждый модуль имеет набор выделенных регистров, в которых хранятся либо какие-нибудь параметры, либо данные о состоянии или некоторая статистика. Между этими регистрами есть глобальное разделение адресного пространства, через которое \texttt{sctrl} и способно доступаться к конкретным модулям.\par
\textbf{Генератор тестовых данных \texttt{patgen}}\par
В целях отладки системы в общей структуре реализован генератор тестовых данных \texttt{patgen}. С его помощью можно осуществлять ввод определяемых пользователем значений АЦП для обработки вместо данных, поступающих от FEB2. Такая возможность используется для тестирования системы и проверки основного функционала независимо от реальных данных с FEB2. Для корректной отладки с помощью \texttt{patgen} в него заложены следующие свойства:\par
\begin{itemize}
    \item данные, генерируемые \texttt{patgen} имеют ту же структуру, что и данные из FEB2;
    \item \texttt{patgen} способен имитировать рассинхронизацию между каналами данных(сдвиг по идентификатору пучка);
    \item каждый канал имеет независимый источник данных;
    \item имеется возможность выбирать между двумя возможными источниками данных(\texttt{patgen} или FEB2) для каждого канала данных в отдельности;
    \item данные генерируются непрерывно циклическим образом, повторение происходит синхронизованно с циклом пучков на орбите.
\end{itemize}\par
Для снижения влияния на процесс компиляции системы целиком в проектирование генератор тестовых сигналов заложен принцип минимизации занимаемых логических ресурсов. В следствие этого, \texttt{patgen} имеет две версии реализации:\par
\begin{itemize}
    \item на основе оперативной памяти: в этой версии используются данные, хранимые во внутренней оперативной памяти ПЛИС, записанные через интерфейс медленного контроля. Такой подход даёт большую гибкость, но занимает большой объём памяти;
    \item на основе функции генерации: в этой версии данные генерируются на лету, используя определённый алгоритм.
\end{itemize}\par
\textbf{Выравниватель входных данных \texttt{ialign}}\par
Первым модулем, который непосредственно принимает входные данные, является \texttt{ialign}. Он предназначен для осуществления выравнивания по времени поступающей информации с FEB2. Входной поток организован в виде кадров, содержащих данные АЦП и два идентификатора столкновения пучков для соответствующих шкал усиления, которые могут быть как идентичными, так и различными. В ходе обработки все данные АЦП выравниваются по одинаковому BCID. При этом порядок оцифрованных значений в рамках каждого отдельного канала может изменяться, однако он не предопределён заранее -- его можно настраивать индивидуально для любого потока, но идентично для парных значений по шкалам усиления.\par
Важная особенность обработки данных модулем \texttt{ialign} -- это расширение данных по временным ячейкам. То есть, по всем каналам с низкого уровня поступает по 6 значений АЦП для каждого идентификатора столкновения пучков, но данный модуль добавляет везде по 2 дополнительных значения с нулевым сигналом корректности, тем самым увеличивая число временных ячеек с данными АЦП до 8. На рисунке \ref{fig:ialign_output} схематично изображён выходной интерфейс компонента. Рабочей тактовой частотой для \texttt{ialign} является $f_{feb}$, соответствующая поступающим с FEB2 данным.

\begin{figure}[ht]
    \centering
    \includegraphics[width=\linewidth]{ialign_output.png}
    \caption{Выходной интерфейс модуля \texttt{ialign}}
    \label{fig:ialign_output}
\end{figure}\par

\textbf{Модуль конфигурируемой перестановки \texttt{remap}}\par
Следующий элемент тракта данных жидкоаргонового сигнального процессора LASP -- модуль \texttt{remap}. Он служит для изменения порядка данных в соответствии с геометрией детектора, ведь в силу ряда технических ограничений, информация от калориметрических ячеек, поступающая через FEB2, находится в перемешанном виде. Путём переупорядочивания данных упрощается задача вычисления сумм энергии ячеек жидкоаргонового калориметра. Такие суммы необходимы для уменьшения полосы пропускания данных в системе fFEX. Как и в случае модуля \texttt{ialign} схема перестановки не является предопределённой -- каждый выходной канал может быть гибко сконфигурирован согласно требованиям. Важной особенностью является то, что помимо всего прочего, компонент конфигурируемой перестановки необходим для реализации перехода данных из тактового домена $f_{feb}$ в домен сигнала $f_{core}$.\par
Сигнальный процессор LASP может иметь одну из двух конфигураций, так называемые медленную и быструю опции. В случае медленной опции \texttt{remap} принимает входной поток данных, состоящий из 88 каналов, в которых содержится по 8 значений АЦП для каждого идентификатора соударения пучка и преобразовывает его в аналогичный поток, но имеющий лишь 64 точно таких же канала. При этом тактовые частоты $f_{feb}$ и $f_{core}$ совпадают по величине 320 МГц, однако могут быть сдвинутыми по фазе. Реальный объём полезных данных не уменьшается, как это может показаться на первый взгляд, поскольку четверть входного трафика составляют значения без сигнала валидности, добавленные модулем \texttt{ialign}, а также присутствуют сигналы, поступающие с неподключенных разъёмов FEB2. В конфигурации быстрой опции та же структура входных данных преобразовывается в 43 выходных канала, каждый из которых имеет целых 12 оцифрованных величин. Поскольку в любом варианте интервал между соседними моментами соударения пучков не изменяется и составляет 25 наносекунд, то в таком режиме тактовая частота выходной шины $f_{core}$ пропорционально увеличена и составляет 480 МГц для обеспечения необходимой плотности данных во времени.\par
\textbf{Ядро обработки данных \texttt{dacore}}\par
Основным обрабатывающим компонентом процессора LASP является ядро обработки данных \texttt{dacore}. Оно преобразовывает поступающие от модуля конфигурируемой перестановки исходные значения АЦП в соответствующие энергетические величины с помощью специальных алгоритмов. Задачи обработки можно разделить на четыре основных функции:\par
\begin{itemize}
    \item определение оптимального коэффициента усиления;
    \item коррекция пьедестала;
    \item вычисление энергии, временной характеристики, а также параметра качества с оптимальным разрешением для системы хранения данных (однако вычисление параметра качества и временной характеристики выполняется только для калориметрических ячеек с выделившейся в них энергией выше заданного порога);
    \item вычисление энергии с уменьшенным разрешением для триггерной системы.
\end{itemize}\par
Следовательно, компонент \texttt{dacore} обеспечивает 2 отдельных выходных потока:\par
\begin{enumerate}
    \item поток для модуля упаковки данных \texttt{packer}, который содержит грубые энергетические значения и флаги превышения порога;
    \item поток для блока буферов, содержащий для каждой калориметрической ячейки энергетическое значение, бит оптимального коэффициента усиления и флаг превышения порога. Для высокоэнергетических ячеек добавляется время импульса и значение качества импульса.
\end{enumerate}\par
Для повышения точности данных, направляемых в систему хранения, используется дополнительная стадия обработки, реализующая алгоритмы цифровой фильтрации. С их помощью достигается восстановление энергии с точностью 1 МэВ, которая затем кодируется многолинейным способом.Для триггерных данных также предусмотрена цифровая фильтрация, предназначенная для подавления шумов и вычисление значений энергии с достаточной точностью для всех модулей принятия триггерных решений, подключенных к задетекторной электронике. Также для этих данных формируется по три бита превышения порогов, количественно описывающие переполнения фонового уровня энергии.\par
\textbf{Процессор онлайн светимости \texttt{olump}}\par
Одним из важнейших показателей работы коллайдера является светимость. Для его расчета в системе жидкоаргонового сигнального процессора предусмотрен модуль процессора онлайн светимости \texttt{olump}. Этот компонент усредняет необработанные оцифрованные значения АЦП, получаемые напрямую с модуля конфигурируемой перестановки \texttt{remap}, по каждому столкновению частиц. Его задачи можно разделить на 4 основных части:\par
\begin{enumerate}
    \item вычисление суммы и суммы квадратов измерений АЦП по шкале высокого коэффициента усиления для настраиваемого набора из 8 каналов. Эти величины вычисляются для каждого столкновения пучков и накапливаются по каждому набору;
    \item буферизация данных АЦП по шкале высокого коэффициента усиления в течение одного полного оборота пучков на орбите Большого Адронного коллайдера. Производится это по тем же наборам каналов, которые были определены выше;
    \item вычисление оценки мгновенной светимости для этих же подмножеств каналов. Эта оценка может быть использована в ядре обработки данных \texttt{dacore} для компенсации влияния светимости на восстановление энергетических и временных величин;
    \item сжатие без потерь значений сумм и сумм квадратов оцифрованных сигналов АЦП.
\end{enumerate}\par
\textbf{Упаковщик триггерных данных \texttt{packer}}\par
Подготовка энергетических значений для их последующей передачи в триггерные системы задетекторной электроники осуществляется силами упаковщика триггерных данных \texttt{packer}. Задачи этого компонента заключаются в следующем:\par
\begin{itemize}
    \item группировка данных, полученных с ядра обработки данных;
    \item кодирование энергий с использованием многолинейного кодирования и их передача в системы глобального триггера и fFEX;
    \item отправка данных в блок буферов;
    \item отправка данных в модуль \texttt{damon}.
\end{itemize}\par
Поток выходных данных для систем глобального триггера и fFEX состоит из кадров, которые содержат информацию о текущем соударении пучков. Помимо этого, в выходном канале требуется отправка служебных кадров, которые не содержат непосредственно полезные данные, а несут различную идентификационную информацию, необходимую, например, для синхронизации.\par
\textbf{Блок буферов \texttt{buffs}}\par
После обработки данные не сразу отправляются в систему хранения, а некоторое время ожидают соответствующего им триггерного сигнала в блоке буферов \texttt{buffs}. Буферизации подлежат все имеющиеся данные: изначальные значения АЦП, обработанные энергетические величины и триггерные данные, полученные от компонентов конфигурируемой перестановки \texttt{remap}, ядра обработки \texttt{dacore} и упаковщика \texttt{packer} соответственно. Время хранения информации требуется не меньшее, чем задержка триггера, которая составляет около 10 мкс.\par
\textbf{Модуль форматирования данных \texttt{fbuild}}\par
Последний этап обработки данных -- формирование из готовых значений фрагментов, пригодных к отправке в FELIX через интерфейс нижнего уровня \texttt{lolli}. Эта задача выполняется с помощью модуля форматирования данных \texttt{fbuild}. Генерируемый формат данных может варьироваться в зависимости от назначения:\par
\begin{itemize}
    \item сбор данных;
    \item калибровка;
    \item отладка;
    \item тестирование системы;
    \item ввод в эксплуатацию.
\end{itemize}\par
Данные, содержащиеся во фрагментах, представляют собой исходные данные АЦП или энергетические значения и связанные с ними биты валидности и качества, а также данные, отправляемые в системы глобального триггера и fFEX. Формат кадра может потребовать отправки определённых или всех этих типов данных. Кроме того, можно выбирать один или несколько потоков выходных данных, хотя обычно используются все.\par
\textbf{Модуль мониторинга данных \texttt{damon}}\par
Кроме системы хранения данных результаты обработки могут передаваться на модуль мониторинга данных \texttt{damon}. Он обеспечивает низкоскоростной канал мониторинга исходных, обработанных и триггерных данных. Эти собранные значения буферизируются до тех пор, пока не будет принято решение о том, отправлять ли их для мониторинга или нет. В конечном итоге, отобранная информация форматируется в Ethernet кадры, которые отправляются на порт XGbE интерфейса нижнего уровня \texttt{lolli}. Компонент \texttt{damon} предполагает реализацию двух возможных режимов работы:\par
\begin{enumerate}
    \item режим мониторинга: в этом режиме осуществляется полный сбор всех входящих данных всех ячеек, которые передаются лишь по определённому условию, например, получению сигнала триггера. Частота передачи этой информации ограничена пропускной способностью внешнего интерфейса(XGbE);
    \item режим прямой трансляции: в этом режиме производится непрерывные сбор и отправка всех входных данных, но лишь для небольшого числа ячеек. Ячейки, которые транслируются в текущий момент, определяются конфигурацией. Количество ячеек, участвующих в режиме трансляции ограничено пропускной способностью внешнего интерфейса(XGbE).
\end{enumerate}\par
\textbf{Монитор состояния аппаратуры \texttt{bomon}}\par
Отдельным модулем, который не является частью тракта обработки данных жидкоаргоновых калориметров детектора ATLAS, однако имеет очень важное значение в функционировании жидкоаргонового сигнального процессора LASP можно выделить монитор состояния аппаратуры \texttt{bomon}. Модуль взаимодействует с устройствами, подключенным к ПЛИС через интерфейс I2C и микросхемой контроллера управления платформой IPMC (Intelligent Platform Management Controller). \texttt{bomon} собирает и передаёт информацию о состоянии внутреннего оборудования ПЛИС LASP, такую как температуру, токи и напряжения, а также считывает информацию с каждого из подключенных электрооптических модулей.\par
Стоит отметить, что представленная на рисунке \ref{fig:lasp} блок схема является не совсем точной, поскольку в реальности структура прошивки жидкоаргонового сигнального процессора более сложная и состоит из целого набора таких систем. Полная схема архитектуры модуля LASP изображена на рисунке \ref{fig:lasp_full}.\par

\begin{figure}[ht]
    \centering
    \includegraphics[width=\linewidth]{lasp_full.png}
    \caption{Полная блок схема архитектуры сигнального процессора LASP}
    \label{fig:lasp_full}
\end{figure}\par

Из схемы видно, что входной поток данных разбивается на 4 независимые части, которые обрабатываются в отдельных подсистемах, изолированно друг от друга. Это сделано благодаря тому, что в данные из каждой части приходят из разных плат FEB2, между которыми не требуется производить перекрёстные операции (напримр, вычислять энергетические суммы по частям калориметра, поступающих на разные модули FEB2). Такой подход позволяет значительно упростить логику каждой подсистемы и улучшить общую разводимость и использование логических ресурсов всей системы на ПЛИС, поскольку разместить несколько меньших независимых модулей проще, чем один большой, выполняющий те же операции. В добавок к описанным модулям в полной структуре добавляется объединитель Ethernet интерфейсов, который преобразует набор поступающих выходных интерфейсов с компонентов мониторинга данных \texttt{damon} в единый интерфейс, который затем отправляется наружу через \texttt{lolli}.\par

%    \subsection{Схема стенда}
%    \input{subsections/Scheme_stand.tex}
    \newpage

%\section{Система на кристалле Xilinx Zynq-7000}
%    \input{sections/ZYNQ.tex}
%    \newpage

\section{Цель и задачи работы}
%    Главной целью данной работы является разработка блока упаковки данных для системы FEX модуля сигнального процессора LASP жидкоаргонового калориметра детектора ATLAS. Эта подсистема состоит из пары связки компонентов, а именно упаковщика данных packer для системы fFEX и  модуля конфигурируемой перестановки remap. То есть по каждому из компонентов необходимо проработать их внутреннюю архитектуру, после чего реализовать на языке описания цифровой логики, что также подразумевает под собой:\par
\begin{itemize}
    \item написание синтезируемых блоков логической аппаратуры;
    \item создание симуляционного окружения и отладка разработанной структуры;
    \item компиляция модулей под целевую платформу;
    \item проверка и оптимизация занимаемых логических ресурсов и временных задержек.
\end{itemize}\par
Помимо работы по непосредственной реализации указанных компонентов сигнального процессора LASP необходимо разработать:\par
\begin{itemize}
    \item программное обеспечение для автоматической генерации конфигураций модуля remap;
    \item формат кадра протокола передачи данных из модуля packer сигнального процессора LASP в систему fFEX.
\end{itemize}\par
В рамках проекта LASP коллаборации ATLAS принято использовать для работы язык описания аппаратуры VHDL. В качестве наиболее возможной потенциальной микросхемы ПЛИС на данный момент рассматривается кристаллы от компании Intel, принадлежащие высокопроизводительному семейству Stratix, а именно Intel Stratix 10 SX 1SX280HU1F50E2VG, являющийся системой на кристалле, имеющей наряду с программируемой логикой также производительный процессор ARM A53 или Intel Stratix 10 MX 1SM21BHU1F53E2VG. Поскольку целевым устройством в любом случае является продукция Intel, то, соответственно, в качестве инструмента разработки ключевую роль занимает программное обеспечение Intel Quartus Prime. Также, в рамках данного проекта применяется симулятор цифровых логических схем QuestaSim.\par

    \newpage

\section{Модуль конфигурируемой перестановки (Remap)}
    Модуль Remap является частью жидкоаргонового сигнального процессора LASP (рис. \ref{fig:remap_lasp}) и в первую очередь предназначен для организации упорядочивания входных данных в соответствии с геометрией детектора.\par
\begin{figure}[ht]
    \centering
    \includegraphics[width=0.8\linewidth]{remap_lasp.png}
    \caption{Схема расположения модуля Remap в общей структуре сигнального процессора LASP}
    \label{fig:remap_lasp}
\end{figure}\par
Remap компонент должен реализовывать приём поступающей информации и формировать из неё набор выходных каналов данных, в каждом из которых обязаны передаваться значения АЦП, выбранные из входных каналов в соответствии с установленной конфигурацией, в определённом порядке, также согласно конфигурации.\par

\subsection{Архитектура модуля}
Основополагающим подходом в проектировании модуля конфигурируемой перестановки Remap является создание таких элементарных устройств, которые способны принимать на вход весь требуемый объём данных и формировать из него лишь один выходной канал. Это обеспечивает высокую гибкость в масштабировании, поскольку в таком случае реализация необходимого количества выходных каналов достигается простой репликацией подобных структур, как это показано на рисунке \ref{fig:remap_replication}.\par

\begin{figure}[ht]
    \centering
    \includegraphics[width=\linewidth]{remap_replication.png}
    \caption{Схема формирования необходимого количества выходных каналов ПЕРЕРИСОВАТЬ}
    \label{fig:remap_replication}
\end{figure}\par

Стоит отметить, поскольку входной поток данных разбивается на 4 независимые части, обрабатывающиеся отдельно, каждый модуль Remap должен принимать лишь 22 канала и формировать 16 или 11 выходов, в зависимости от варианта сигнального процессора LASP.\par
В ходе разработки было спроектировано и реализовано два различных варианта архитектуры базового элемента модуля Remap: с модулем синхронизации тактовых доменов, а также архитектура, основанная на FIFO памяти.\par

\subsubsection{Архитектура с модулем синхронизации тактовых доменов}
Первый вариант архитектуры компонента Remap, содержащий специальный модуль синхронизации тактовых доменов, представлен на рисунке \ref{fig:remap_cds}.\par
\begin{figure}[ht]
    \centering
    \includegraphics[width=\linewidth]{remap_cds.png}
    \caption{Схема архитектуры модуля Remap с модулем синхронизации тактовых доменов ПЕРЕВЕСТИ}
    \label{fig:remap_cds}
\end{figure}\par
Основной особенностью этой архитектуры является то, что в качестве буфера для мультиплексированных данных используется блок двухпортовой RAM памяти. Эта память разбита на несколько страниц, каждая из которых имеет размер, достаточный для хранения захваченной информации, относящейся к одному столкновению пучков. Чтение данных из страницы начинается лишь только после её полного заполнения записывающей стороной. Для синхронизации процессов считывания и записи предусмотрен следующий механизм: по завершению заполнения страницы памяти записывающая логика генерирует импульс шириной в один такт и отправляет его на вход специального модуля. Внутри этого модуля расположены два счётчика, работающие на тактовых частотах $f_{feb}$ и $f_{core}$, которые ведут счёт в диапазоне количества временных ячеек для каждого BCID. В рабочем режиме первый настроен так, чтобы обнуляться одновременно с поступлением синхросигнала от записывающей логики, а второй с задержкой около такта $f_{feb}$ после первого. При завершении цикла работы второго счётчика формируется выходной сигнал синхронизации, который поступает к считывающей логике и означает, что очередная страница в двухпортовой памяти заполнена полностью и можно безопасно извлекать из неё данные. Если вдруг синхронизация собьётся и синхросигнал от системы записи придёт не вовремя, то модуль это обнаружит и перейдёт в режим восстановления синхронизации. Часть данных после сбоя синхронизации будет утеряно, но через некоторое время система автоматически восстановится и продолжит работать исправно.\par
Описанная архитектура была реализована на языке описания цифровой логики VHDL и отлажена. По результатам тестирования в симуляторе она подтвердила свою работоспособность. Однако такой подход имеет ряд недостатков, главным из которых является необходимость передавать целый набор сигналов(такие как номер текущей страницы, индекс столкновения пучка, а также ряд вспомогательных сигналов внутри модуля синхронизации тактовых доменов) между тактовыми доменами $f_{feb}$ и $f_{core}$ вручную, используя схемы на двух регистрах. Для корректной организации таких переходов требуется тонкая ручная настройка временных ограничений, реализуемая путём составления специальных указаний синтезатору физической схемы, входящему в состав программного комплекса Intel Quartus Prime. Это значительно усложняет весь проект и делает его гораздо менее гибким. После возникновения проблем с разводимостью логики проекта LATOME, который является основой задетекторной электроники эксперимента ATLAS, разработанной в рамках предшествующей фазы обновления детектора, командой разработчиков сигнального процессора LASP было принято решение максимально избегать подобные способы перехода между тактовыми доменами. Кроме того, данный вариант является довольно путанным и сложным для понимания в деталях. Учитывая все эти недостатки, было решено разработать альтернативную архитектуру модуля Remap.\par


\subsubsection{Архитектура, основанная на FIFO}
Второй вариант архитектуры компонента Remap, содержащий память FIFO, представлен на рисунке \ref{fig:remap_fifo}.\par
\begin{figure}[ht]
    \centering
    \includegraphics[width=\linewidth]{remap_fifo.png}
    \caption{Схема архитектуры модуля Remap, основанной на FIFO ПЕРЕВЕСТИ}
    \label{fig:remap_fifo}
\end{figure}\par



\subsection{Конфигурирование через интерфейс медленного контроля}
Конфигурирование модуля Remap осуществляется через интерфейс медленного контроля. Как упоминалось ранее, он функционирует поверх протокола Avalon Memory Mapped, который предназначен для работы с адресуемой памятью. Такой подход очень удобен, поскольку в этом случае можно выделить каждому модулю свой участок адресов, по которым можно будет располагать необходимые значения. Разные адреса можно настроить по способу доступа к ним, таким образом можно завести некоторые показатели системы, которые можно будет только считывать, или же добавить параметры с опцией модификации. Отдельная важная особенность работы через память -- возможность функционирования в разных тактовых доменах, для этого достаточно использовать модули двухпортовой памяти. Это позволяет использовать достаточно низкую тактовую частоту для интерфейса конфигурации, чтобы он не оказывал существенного влияния на разводимость остальной логики. Причём эта частота может быть единой для конфигурирования всех компонентов, вне зависимости от их внутренних тактовых сигналов, что значительно упрощает работу медленного контроля.\par
Модуль перестановки Remap имеет две конфигурируемые стадии: какие значения извлекать из общего потока данных с помощью мультиплексора и в каком порядке их выдавать в выходной канал. Поскольку эти стадии работают в разных тактовых доменах, то необходимо размещать параметры для них в разных блоках памяти, чтобы можно было корректно переводить значения в целевые тактовые частоты. Начальный адрес конфигурации мультиплексора устанавливается глобальной константой REMAP\_BADDR(Remap Base Address) с уровня всего проекта сигнального процессора LASP, а конфигурация порядка выходных данных имеет некоторое смещение относительно него. На рисунке \ref{fig:remap_sctrl_mapping} изображена схема отображения конфигураций на адресное пространство.\par
\begin{figure}[ht]
    \centering
    \includegraphics[width=\linewidth]{remap_sctrl_mapping.png}
    \caption{Схема маппинга памяти модуля перестановки Remap для записи конфигурации}
    \label{fig:remap_sctrl_mapping}
\end{figure}\par
Для конфирурирования входного мультиплексора необходимо для каждой временной ячейки установить номер канала, с которого необходимо захватить данные. На каждый Remap поступает по 22 канала, то есть требуется 5 бит на значение. Для любого столкновения пучков выделяется по 8 временных интервалов, следовательно суммарно должно быть не менее 40 бит данных для конфигурирования одного выходного канала Remap. Шина данных интерфейса Avalon Memory Mapped имеет ширину 32 бита, поэтому для удобства формирования и чтения конфигурационных данных используется 2 слова AVMM, что составляет 64 бита. В случае варианта быстрой опции сигнального процессора LASP требуется два входных мультиплексора, соответственно размер конфигурации удваивается и равняется 128 бит.\par
Конфигурирование финальной перестановки осуществляется путём последовательного указания индекса необходимого значения. В зависимости от медленной или быстрой опции отобранных величин может быть 8 или 16 соответственно. Для более удобной работы под каждое такое значение выделяется по 4 бита. Далее, в зависимости от варианта сигнального процессора LASP требуется от 8 до 12 временных ячеек для каждого BCID, следовательно суммарно необходимо иметь от 32 до 48 бит. Аналогично конфигурации мультиплексора, в целях повышения удобства размер конфигурации округляется по ширине шины интерфейса AVMM и составляет 64 бита независимо от опции сигнального процессора LASP.\par


\subsection{Реализация}
В ходе реализации синтезируемых компонентов модуля Remap активно использовалось тестирование с помощью симуляции. Оно осуществлялось с помощью специализированного программного обеспечения Mentor QuestaSim, предназначенное для моделирования и отладки микросхем ПЛИС. Симуляционное окружение разработано, как и синтезируемые модули, на языке VHDL и обеспечивает поступление данных на входной интерфейс тестируемого модуля. Так, на рисунке \ref{fig:sim_input} приведён фрагмент симуляции, на котором показан пример данных внутри внутри  входного интерфейса. Можно увидеть, что как и в реальной системе, в каждый модуль Remap поступает 22 канала со значениями АЦП, причем для каждого BCID передаётся по 8 величин в канале. Все сигналы входного сигнала синхронны с тактовой частотой $f_{feb}$.\par
\begin{figure}[ht]
    \centering
    \includegraphics[width=\linewidth]{sim_input.png}
    \caption{Фрагмент поступающих в модуль Remap входных данных в симуляции ВЗЯТЬ МАСШТАБ ПОКРУПНЕЙ}
    \label{fig:sim_input}
\end{figure}\par
В рассматриваемом примере модуль предназначен для работы в варианте сигнального процессора LASP с установленной медленной опцией. В качестве конфигурации производится установка параметров для первых двух выходных каналов Remap компонента. На рисунке \ref{fig:sim_sctrl} отображено, как это осуществляется через интерфейс медленного контроля. На волновой диаграмме отчетливо видно, как значения поступают в установленном формате по протоколу AVMM, после чего лишние биты отсекаются, а сами конфигурационные данных переходят в соответствующие им тактовые домены. В соответствии с настройкой, первый выходной канал должен выдавать данные из первых восьми входных каналов в обратной последовательности, а второй по четыре значения из каналов с номерами 20 и 21 в чередующейся последовательности.\par
\begin{figure}[ht]
    \centering
    \includegraphics[width=\linewidth]{sim_sctrl.png}
    \caption{Пример записи конфигурации модуля Remap в симуляции}
    \label{fig:sim_sctrl}
\end{figure}\par
На рисунке \ref{fig:sim_output} изображен выходной интерфейс модуля конфигурируемой перестановки. Поскольку система предназначена для работы в медленной опции сигнального процессора LASP, выходной интерфейс состоит из 16 каналов, в котором данные передаются синхронно частоте $f_{core}$, равной 320 МГц. На нём можно отследить корректность работы компонента, работающего в соответствии с вышеописанными настройками. \par
\begin{figure}[ht]
    \centering
    \includegraphics[width=\linewidth]{sim_output.png}
    \caption{Фрагмент выходящих из модуля Remap данных в симуляции ВЗЯТЬ МАСШТАБ ПОКРУПНЕЙ}
    \label{fig:sim_output}
\end{figure}\par


%    \subsection{Процессорная система}
%    \input{subsections/PS.tex}
%    \subsection{Программируемая логика}
%    \input{subsections/PL.tex}
    \newpage

\section{Модуль упаковки данных (Packer)}
    Модуль \texttt{packer} является частью сигнального процессора LASP и предназначен для вычисления необходимых энергетических сумм по узлам калориметра и последующей упаковки данных для отправки в целевую систему.\par
\begin{figure}[ht]
    \centering
    \includegraphics[width=0.8\linewidth]{packer_lasp.png}
    \caption{Схема расположения модуля packer в общей структуре сигнального процессора LASP}
    \label{fig:packer_lasp}
\end{figure}\par

Модуль \texttt{packer} обеспечивает данные для двух внешних подсистем: глобальный триггер и fFEX. Система fFEX \parencite{tdr_blue} является представителем набора модулей FEX, с помощью которых осуществляется поиск специфичных событий в ускорителе. Для этого ей не требуется полный объём данных, поступающих с детектора, а достаточно лишь определённой части, причём зачастую используются не конкретные значения, а суммы по целым участкам калориметра.\par
Для передачи данных во внешние подсистемы используется подход упаковки информации в кадры, которые уже непосредственно отправляются клиенту. Поскольку в системе сбора данных имеется жесткая система синхронизации -- приёмник и передатчик синхронизованы между собой и частотой БАК -- при передаче данных не требуется синхронизация в каждом пакете. Полоса пропускания полностью используется для передачи данных, упакованных в кадры, в каждом из которых содержатся данные от одного столкновния пучков. На рисунке \ref{fig:packer_ffex_frame} изображен один из предложенных вариантов формата кадра данных. Он содержит 46 десятибитных значений АЦП, идентификатор столкновения пучков, а также флаги превышения данными порога $2 \sigma$. Помимо представленного варианта также были предложения реализовать кадры с переменной структурой, которые наиболее эффективно использовали бы пропускную способность канала передачи данных для разных участков калориметра.\par
\begin{figure}[ht]
    \centering
    \includegraphics[width=\linewidth]{packer_ffex_frame.png}
    \caption{Вариант формата кадра данных для отправки в систему fFEX}
    \label{fig:packer_ffex_frame}
\end{figure}\par
В ходе работы был разработан прототип модуля \texttt{packer} на языке описания цифровой логики VHDL, и подготовлено тестовое симуляционное окружение для моделирования его поведения. Также этот модуль был интегрирован в общий дизайн сигнального процессора LASP в рамках подхода top-down \parencite{topdown}. Таким образом, весь дизайн целиком LASP сейчас может модулироваться и компилироваться. Целевая функциональность модуля упаковки триггерных данных будет реализована после согласования требований с разработчиками системы fFEX (ожидается в 2023).\par

\newpage

%\section{Веб-сервер}
%    \input{sections/Server.tex}
%    \subsection{Серверная часть}
%    \input{subsections/Server_part.tex}
%    \subsection{Клиентская часть}
%    \input{subsections/Client_part.tex}
%    \newpage

\section*{Заключение}
\addcontentsline{toc}{section}{Заключение}
    В рамках данной работы велась разработка блока упаковки данных сигнального процессора жидкоаргоновых калориметров LASP для системы FEX, состоящей из связки модулей конфигурируемой перестановки \texttt{remap} и упаковщика триггерных данных \texttt{packer}. Таким образом, были реализованы следующие задачи:\par
\begin{itemize}
    \item по модулю конфигурируемой перестановки \texttt{remap}:
        \begin{itemize}
            \item проработана внутренняя архитектура -- составлено 2 альтернативных варианта;
            \item написаны синтезируемые блоки цифровой логики на языке VHDL;
            \item создано симуляционное окружение для моделирования поведения модуля;
            \item проведена компиляция под целевую платформу;
            \item выполнена интеграция модуля в основную структуру сигнального процессора LASP;
            \item разработан прототип программного обеспечения для автоматической генерации конфигурации;
        \end{itemize}
    \item по упаковщику триггерных данных \texttt{packer} для системы fFEX:
        \begin{itemize}
            \item разработаны протоколы упаковки данных в кадры;
            \item реализован синтезируемый прототип модуля;
            \item создано симуляционное окружение для моделирования поведения модуля;
            \item выполнена интеграция модуля в основную структуру сигнального процессора LASP.
        \end{itemize}
\end{itemize}\par
В дальнейшем планируется масштабирование программного обеспечения для автоматической генерации конфигураций модуля \texttt{remap} на все участки жидкоаргонового калориметра. После согласования требований с разработчиками системы fFEX будет завершена реализация упаковщика \texttt{packer}. После завершения разработки предстоит запуск систем на целевой платформе и последующий ввод в эксплуатацию.\par

    \newpage

%\inputencoding{T2A}
%\usepackage[koi8-r]{inputenc}
%\bibliographystyle{utf8gost705u}
\printbibliography
%\printbibliography{sections/bibliography}
%\inputencoding{utf8}

\addcontentsline{toc}{section}{Список литературы}
\end{document}
